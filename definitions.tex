\usepackage{amsmath}
\usepackage{amssymb}
%\usepackage{amsthm}
\usepackage{amsfonts}
\usepackage{dsfont}
%\usepackage{pifont}% http://ctan.org/pkg/pifont
\newcommand{\cmark}{\ding{52}}%
\newcommand{\xmark}{\ding{56}}%
\usepackage{etoolbox}	% robustify command

%%%%%%%%%%%%%%%%%%%%%%%%%%%%%%%%%%%%%%%%%%%%%%%%%%%%%%%%%%%%%%%%%%%%%%%%%%%%%%%%
% General utilities
%%%%%%%%%%%%%%%%%%%%%%%%%%%%%%%%%%%%%%%%%%%%%%%%%%%%%%%%%%%%%%%%%%%%%%%%%%%%%%%%

% etal command
\newcommand{\etal}{\emph{et al.}\ }

%%%%%%%%%%%%%%%%%%%%%%%%%%%%%%%%%%%%%%%%%%%%%%%%%%%%%%%%%%%%%%%%%%%%%%%%%%%%%%%%
% Probability
%%%%%%%%%%%%%%%%%%%%%%%%%%%%%%%%%%%%%%%%%%%%%%%%%%%%%%%%%%%%%%%%%%%%%%%%%%%%%%%%

\newcommand{\prob}[1]{\mathrm{P}\left( #1 \right)}

% Gaussian
\newcommand{\DistributionGaussian}[2]{\mathcal{N}(#1,#2)}

% Noisy format (tilde)
\newcommand{\noisy}[1]{\tilde{#1}}

% Estimate (hat upper)
\newcommand{\estimate}[1]{\hat{#1}}

% Mahalanobis norm
\newcommand{\mahalanobisNorm}[2]{\lVert{#1}\rVert^{2}_{#2}}

% Huber norm
\newcommand{\huberNorm}[2]{\rho_h\left(#1\right)_{#2}}

% Covariance of (text version)
\newcommand{\Cov}[1]{\mathrm{Cov}\!\left(#1\right)}

%%%%%%%%%%%%%%%%%%%%%%%%%%%%%%%%%%%%%%%%%%%%%%%%%%%%%%%%%%%%%%%%%%%%%%%%%%%%%%%%
% Optimization
%%%%%%%%%%%%%%%%%%%%%%%%%%%%%%%%%%%%%%%%%%%%%%%%%%%%%%%%%%%%%%%%%%%%%%%%%%%%%%%%

% Optimum notation (superscript asterisk)
\newcommand{\optimum}[1]{{#1}^{*}}

% partial derivative
\newcommand{\diffPartial}[2]{\displaystyle \frac{\partial #1}{\partial #2}}

% argmin
\newcommand{\argmin}{\operatornamewithlimits{arg\,min}}
% argmax
\newcommand{\argmax}{\operatornamewithlimits{arg\,max}}

%%%%%%%%%%%%%%%%%%%%%%%%%%%%%%%%%%%%%%%%%%%%%%%%%%%%%%%%%%%%%%%%%%%%%%%%%%%%%%%%
% Geometry
%%%%%%%%%%%%%%%%%%%%%%%%%%%%%%%%%%%%%%%%%%%%%%%%%%%%%%%%%%%%%%%%%%%%%%%%%%%%%%%%

% Euclidean space
\newcommand{\Rn}[1]{\mathbb{R}^{#1}}

% Trace
\newcommand{\traceNew}[1]{\mathrm{tr}(#1)}

% skew symmetric matrix
\newcommand{\matrixSkew}[3]{
	\begin{bmatrix}
		  0 & -#3 &  #2 \\
		 #3 &   0 & -#1 \\
		-#2 &  #1 &   0
	\end{bmatrix}
	}

% Transformation Matrix
\newcommand{\matrixRigidBody}[2]{
	\left[
	\begin{array}{cc}
		#1  &  #2 \\
		0_{1\times3} &   1
	\end{array}
	\right]
}

% Extrinsic Matrix
\newcommand{\matrixExtrinsic}[2]{
	\left[
	\begin{array}{c|c}
		#1 & #2
	\end{array}
	\right]
}

% camera projection model
\newcommand{\cameraProjectionModel}[2]{
	\notatVector{\pi}({#1}, {#2})
}

% camera depth map model (LSD SLAM)
\newcommand{\cameraDepthMapModel}[3]{
	\notatVector{\pi}({#1}, {#2}, {#3})
}

% inverse camera projection model
\newcommand{\inverseCameraProjectionModel}[3]{
	\notatVector{\pi}^{-1}({#1}, {#2}, {#3})
}



% Frame
% subarrow used in the frame notation
\newcommand{\subarrow}[1]{
	\mathord{
		\renewcommand{\arraystretch}{0}
		\begin{array}[t]{@{}c@{}l@{}}
			#1\\[2pt]
			\hspace{-2pt}\scriptstyle\longrightarrow
		\end{array}
		\kern\scriptspace
	}
}
% frame definition
\newcommand{\notatFrame}[1]{\subarrow{\mathcal{F}}{}_{\scriptscriptstyle #1}}

% Format for matrices, vectors, scalars, homogeneous points and manifolds
% Single letters
\newcommand{\notatMatrix}[1]{\boldsymbol{\mathrm{#1}}}
\newcommand{\notatVector}[1]{\boldsymbol{\mathrm{#1}}}
\newcommand{\notatScalar}[1]{{#1}}
\newcommand{\notatHomog}[1]{\boldsymbol{{#1}}}
\newcommand{\notatManifold}[1]{\mathcal{\MakeUppercase{#1}}}

% Letters with right subscript
\newcommand{\notationMatrix}[2]{\boldsymbol{\mathrm{#1}}_{\scriptscriptstyle #2}}
\newcommand{\notationVector}[2]{\boldsymbol{\mathrm{#1}}_{\scriptscriptstyle #2}}
\newcommand{\notationScalar}[2]{{#1}_{\scriptscriptstyle #2}}
\newcommand{\notationHomog}[2]{\boldsymbol{{#1}}_{\scriptscriptstyle #2}}
\newcommand{\notationManifold}[2]{\mathcal{\MakeUppercase{#1}}_{\scriptscriptstyle #2}}

% Letters with left and right subscript
\newcommand{\notationMatrixFrame}[3]{{\scriptscriptstyle_#2}\boldsymbol{\mathrm{#1}}_{\scriptscriptstyle #3}}
\newcommand{\notationVectorFrame}[3]{{\scriptscriptstyle_#2} \boldsymbol{ \mathrm{#1}}_{\scriptscriptstyle #3}}
\newcommand{\notationScalarFrame}[3]{{\scriptscriptstyle_#2}{#1}_{\scriptscriptstyle #3}}
\newcommand{\notationHomogFrame}[3]{{\scriptscriptstyle_#2}\boldsymbol{{#1}}_{\scriptscriptstyle #3}}

% robustify enables to use the previous definitions within captions and stuff
\robustify{\notatFrame}
\robustify{\notatMatrix}
\robustify{\notatVector}
\robustify{\notatScalar}
\robustify{\notatHomog}
\robustify{\notationMatrix}
\robustify{\notationVector}
\robustify{\notationScalar}
\robustify{\notationHomog}
\robustify{\notationMatrixFrame}
\robustify{\notationVectorFrame}
\robustify{\notationScalarFrame}
\robustify{\notationHomogFrame}

%%%%%%%%%%%%%%%%%%%%%%%%%%%%%%%%%%%%%%%%%%%%%%%%%%%%%%%%%%%%%%%%%%%%%%%%%%%%%%%%
% Lie Groups
%%%%%%%%%%%%%%%%%%%%%%%%%%%%%%%%%%%%%%%%%%%%%%%%%%%%%%%%%%%%%%%%%%%%%%%%%%%%%%%%

% Lie Groups
\newcommand{\hatop}[1]{#1^{\wedge}}
\newcommand{\veeop}[1]{#1^{\vee}}

\newcommand{\liebracket}[2]{\left[ #1, #2\right]}

% GL(n)
\newcommand{\GLN}{\mathrm{GL(N)}}

% SO(2)
\newcommand{\sotwo}{\mathfrak{so}(2)}
\newcommand{\SOtwo}{\mathrm{SO(2)}}

% SO(3)
\newcommand{\sothree}{\mathfrak{so}(3)}
\newcommand{\SOthree}{\mathrm{SO(3)}}

% SO(N)
\newcommand{\soN}{\mathfrak{so}(N)}
\newcommand{\SON}{\mathrm{SO(N)}}

% SE(3)
\newcommand{\sethree}{\mathfrak{se}(3)}
\newcommand{\SEthree}{\mathrm{SE(3)}}

% SE(N)
\newcommand{\seN}{\mathfrak{se}(N)}
\newcommand{\SEN}{\mathrm{SE(N)}}

% Sim(3)
\newcommand{\simthree}{\mathfrak{sim}(3)}
\newcommand{\Simthree}{\mathrm{Sim(3)}}

% Generic exponential and logarithm map (using the capitalized version of Forster et al. (2015))
\newcommand{\Expmap}[1]{\mathrm{Exp}\left(#1\right)}
\newcommand{\expmap}[1]{\mathrm{exp}\left(#1\right)}
\newcommand{\Logmap}[1]{\mathrm{Log}\left(#1\right)}
\newcommand{\logmap}[1]{\mathrm{log}\left(#1\right)}

% SO(3) exponential and logarithm maps
\newcommand{\ExpmapSOthree}[1]{\mathrm{Exp}_{\SOthree}\left(#1\right)}
\newcommand{\expmapSOthree}[1]{\mathrm{exp}_{\SOthree}\left(#1\right)}
\newcommand{\LogmapSOthree}[1]{\mathrm{Log}_{\SOthree}\left(#1\right)}
\newcommand{\logmapSOthree}[1]{\mathrm{log}_{\SOthree}\left(#1\right)}

% SE(3) exponential and logarithm maps
\newcommand{\ExpmapSEthree}[1]{\mathrm{Exp}_{\SEthree}\left(#1\right)}
\newcommand{\expmapSEthree}[1]{\mathrm{exp}_{\SEthree}\left(#1\right)}
\newcommand{\LogmapSEthree}[1]{\mathrm{Log}_{\SEthree}\left(#1\right)}
\newcommand{\logmapSEthree}[1]{\mathrm{log}_{\SEthree}\left(#1\right)}

% Generic adjoint
\newcommand{\adjop}[1]{\mathrm{ad}\left(#1\right)}
\newcommand{\Adjop}[1]{\mathrm{Ad}\left(#1\right)}

% Generic Right and Left jacobian
\newcommand{\JacR}[1]{\mathit{J}_{r} \left( #1 \right)}
\newcommand{\JacInvR}[1]{\mathit{J}_{r}^{-1} \left( #1 \right)}
\newcommand{\JacL}[1]{\mathit{J}_{l} \left( #1 \right) }
\newcommand{\JacInvL}[1]{\mathit{J}_{l}^{-1} \left( #1 \right)}

% Barfoot's operators (Barfoot & Furgale, 2014)
\newcommand{\barfootOpA}[1]{\langle\!\langle #1 \rangle\!\rangle}
\newcommand{\barfootOpAB}[2]{\langle\!\langle #1, #2 \rangle\!\rangle}

\newcommand{\adjhat}[1]{{#1}^{\curlywedge}}
\newcommand{\adjvee}[1]{{#1}^{\curlyvee}}

%%%%%%%%%%%%%%%%%%%%%%%%%%%%%%%%%%%%%%%%%%%%%%%%%%%%%%%%%%%%%%%%%%%%%%%%%%%%%%%%
% Other stuff
%%%%%%%%%%%%%%%%%%%%%%%%%%%%%%%%%%%%%%%%%%%%%%%%%%%%%%%%%%%%%%%%%%%%%%%%%%%%%%%%

% image intensity norm
\newcommand{\imageIntensity}[2]{I_{#1} \left( #2 \right)}

\newcommand{\getZ}[1]{\boldsymbol{\mathsf{Z}}\left(#1\right)}

% matrix spacing adjustments
% usage: 
% \begin{pmatrix}[1.5]
% ...
% \end{pmatrix}

\makeatletter
\renewcommand*\env@matrix[1][\arraystretch]{%
	\edef\arraystretch{#1}%
	\hskip -\arraycolsep
	\let\@ifnextchar\new@ifnextchar
	\array{*\c@MaxMatrixCols c}}
\makeatother

% table stuff
\newcommand{\cell}[1]{\begin{tabular}{@{}l@{}}#1\end{tabular}}

% colors
\usepackage{color}
\usepackage{colortbl}
\definecolor{ColorLightCyan}{rgb}{0.88,1,1}
\definecolor{ColorLightTurquoise}{rgb}{0.5, 1, 0.8}
