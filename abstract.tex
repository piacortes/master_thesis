%\begin{resumen}


%\end{resumen}

\begin{abstract}
The Transit Monitoring in the South (TraMoS) project aims to perform photometric follow-up in known transiting exoplanets. This process provides not only essential information about the transiting planet but also about their planetary system's architecture. Departures from the linear ephemeris given by Keplerian motion, the so-called Transit Timing Variations (TTV), can only be spotted through extensive follow-up and could suggest the existence of additional bodies in the system, such as planetary companions or exomoons. 

This Master thesis recapitulates the last years of the TraMoS project, on which new targets were selected, modern tools were used and new analysis was performed. I studied 27 new light curves of three hot Jupiters: WASP-18b, WASP-19b, and WASP-77Ab. Their orbital and physical parameters were refined, and their linear ephemeris updated thanks to the inclusion of archival transit time data. Given the lack of significant variations in their transit times, dynamical analysis can place upper mass limits in hypothetical perturbers. Bodies with masses greater than $50\,M_{\odot}$ can be discarded for WASP-18b, $6\,M_{\odot}$ for WASP-19b, and $10\,M_{\odot}$ for WASP-77Ab. Furthermore, I included recent TESS data for the three targets, supporting the linear ephemeris as the best fit for their transit times

The results presented in this thesis support the rare occurrence of companions in hot Jupiters systems. Further studies may confirm if this kind of planet goes through a different formation process ending up orbiting alone on their systems or they have very small and unseen companions.

\end{abstract}

