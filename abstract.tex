\begin{resumen}
El proyecto Transit Monitoring in the South (TraMoS) busca realizar seguimiento fotométrico de exoplanetas transitantes conocidos. Este proceso provee no sólo información acerca del planeta transitante, sino también sobre la arquitectura del sustema planetario. Desviaciones de la efemeris lineal producida por el movimiento Kepleriano, conocidas como Transit Timing Variations (TTV), sólo se pueden detectar mediante intensivo seguimiento y pueden sugerir la existencia de cuerpos adicionales en el sistema, tales como compañeros planetarios o exolunas.

Esta tesis de magíster recapitula los últimos años del proyecto TraMoS, se seleccionaron nuevos objetivos, se utilizaron nuevas heramientas, y se desarrollaron nuevos analisis. Se estudiaron 27 nuevas curvas de luz de tres Jupiter calientes: WASP-18b, WASP-19b, y WASP-77Ab. Se refinaron sus parámetros físicos y orbitales, y la ecuación de efemeris lineal fue actualizada gracias a la inclusión de datos de transito históricos. Debido a la carencia de variaciones significativas en sus tiempos central de tránsito, la realización de análisis dinámicos puede imponer límites superiores a la masa de perturbadores hipotéticos. Se descartaron cuerpos que poseen masas más grandes que $50\,M_{\oplus}$ para WASP-18b, $6\,M_{\oplus}$ para WASP-19b, y $10\,M_{\oplus}$ para WASP-77Ab. Adicionalmente, se incluyeron datos de TESS para los tres objetivos, lo que apoya que la efemeris lineal es el mejor modelo para los tiempos de tránsito.

Los resultados presentados en esta tesis apoyan la poco frecuente aparición de compañeros en sistemas de Júpiters calientes. Estudios futuros podrían evidenciar si estos tipos de planeta sufrieron un proceso de formación diferente por el cual terminan orbitando solos, o si efectivamente poseen compañeros enanos que no se han detectado.
\end{resumen}

\begin{abstract}
The Transit Monitoring in the South (TraMoS) project aims to perform photometric follow-up in known transiting exoplanets. This process provides not only essential information about the transiting planet but also about their planetary system's architecture. Departures from the linear ephemeris given by Keplerian motion, the so-called Transit Timing Variations (TTV), can only be spotted through extensive follow-up and could suggest the existence of additional bodies in the system, such as planetary companions or exomoons. 

This Master thesis recapitulates the last years of the TraMoS project, on which new targets were selected, modern tools were used and new analysis was performed. I studied 27 new light curves of three hot Jupiters: WASP-18b, WASP-19b, and WASP-77Ab. Their orbital and physical parameters were refined, and their linear ephemeris updated thanks to the inclusion of archival transit time data. Given the lack of significant variations in their mid-transit times, dynamical analysis can place upper mass limits in hypothetical perturbers. Bodies with masses greater than $50\,M_{\oplus}$ can be discarded for WASP-18b, $6\,M_{\oplus}$ for WASP-19b, and $10\,M_{\oplus}$ for WASP-77Ab. Furthermore, I included recent TESS data for the three targets, supporting the linear ephemeris as the best fit for their transit times.

The results presented in this thesis support the rare occurrence of companions in hot Jupiter systems. Further studies may confirm if this kind of planet goes through a different formation process ending up orbiting alone on their systems or they have very small and unseen companions.

\end{abstract}

