%\begin{resumen}


%\end{resumen}

\begin{abstract}
The Transit Monitoring in the South (TraMoS) project aims to perform photometric follow-up in known transiting exoplanets. Trough photometric follow-up not only essential information about the transiting planet is provided, but their planetary system's architecture. Departures from the linear ephemeris given by Keplerian motion can only be spotted through extensive follow-up, these phenomena are called Transit Timing Variations (TTV) and could suggest the existence of additional bodies in the system, such as planetary companions or exomoons. 

This Master thesis recapitulates the last years of the TraMoS project, on which new targets were selected, new tools were used and new analysis were performed. I studied 27 new light curves of three hot Jupiters: WASP-18b, WASP-19b, and WASP-77Ab. Their orbital and physical parameters were refined, and their linear ephemeris updated thanks to the inclusion of archival transit time data. Given the lack of significant variations in their transit times, bodies with masses greater than $50\,M_{\odot}$ can be discarded for WASP-18b, $6\,M_{\odot}$ for WASP-19b, and $10\,M_{\odot}$ for WASP-77Ab. After the inclusion of the recent TESS data for the three targets, the linear ephemeris remains as the best fit for their transit times. These results support the rare existence of companions in hot Jupiters systems.
\end{abstract}

