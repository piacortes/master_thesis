\chapter{Observations and Data Reduction}\label{chap:obs}

We collected 8 light curves for WASP-18b between 2009 and 2017, 9 light curves for WASP-19b between 2011 and 2017, and 5 light curves for WASP-77Ab between 2015 and 2017. We included 4 transits of WASP-77Ab from the Exoplanet Transit Database (ETD) in order to cover a larger timespan.

All of the photometry are collected by using either the Danish 1.54 m telescope at ESO La Silla Observatory, or the SMARTS 1 m at Cerro Tololo Observatory (CTIO), except for one transit of WASP-77Ab that was observed with the Warsaw 1.3 m at Las Campanas Observatory (LCO). The log of our observations is shown in Table XXX. All the new light curves used for this work are presented in Figure XXX.

For the photometric observations conducted on the Danish telescope, we used the Danish Faint Object Spectrograph and Camera (DFOSC) instrument, which has a $2{\rm K} \times 2{\rm K}$ CCD with a 13.7 x 13.7 arcmin$^2$ field of view (FoV) and a pixel scale of 0.39" per pixel. To reduce the readout time, some of the Danish 1.54 m images were windowed to only include the target star and its closest reference stars. The observations of the transits of WASP-18b during 2016 and 2017 were forced to be windowed due to a malfunction of the CCD. For those transits, only one reference star was used to perform the photometry.

The SMARTS 1 m has the Y4KCam instrument which is a $4{\rm K} \times 4{\rm K}$ CCD camera with a $20\times20$ arcmin$^2$ FoV and a pixel scale of 0.289" per pixel. 

For the observation on the Warsaw 1.3 m telescope, we used a $2048 \times 4096$ CCD camera chip with a 1.4 square degrees of FoV and 0.26" per pixel scale. No windowing or binning was used during the observations on both SMARTS 1m and the Warsaw 1.3m telescope.

As suggested by \cite{Southworth2009}, most of our observation, especially those conducted after 2011, used the defocus technique, which allows longer exposure times in bright targets and improves the photometric precision. We adjust the exposure time during the observations if the weather is not ideal. The recorded Julian Date in the Coordinated Universal Time (${\rm JD_{UTC}}$) were converted into Barycentric Julian Date in the Barycentric Dynamical Time standard (${\rm BJD_{TDB}}$) by following the procedure as in \citet{Eastman2010}.

We reduced the data by using our custom pipeline. It follows the standard procedures of reduction, calibration, and aperture photometry, but customized for each used instrument. The pipeline semi-automatically finds the best aperture and ring size, for the sky that produces the light curve with less RMS. Then, we manually choose the reference stars to produce the differential light curves for each targets.