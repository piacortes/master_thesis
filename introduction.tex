\chapter{Introduction}\label{chap:intro}
The existence of planets orbiting other stars different from our Sun lived in the human imagination for centuries. During the sixteenth century, the Italian philosopher Giordano Bruno suggested, for the first time in history, that more planets could be outside the Solar System. Even though the first claims of exoplanet detection began at the end of the nineteenth century, it was not until more than four hundred years after Bruno's statement, that the first exoplanet was confirmed in 1992. Surprisingly, it was not just one exoplanet but three, orbiting the pulsar PSR B1257+12 more than 1000 light-years away from us. Today, more than 4,000 exoplanets are confirmed, and thousands more are waiting for their confirmation.

Having billions of billions of galaxies in the Universe each one with millions of stars, the current number of discovered exoplanets is still very low. Furthermore, the hunting zone of exoplanets is limited by our current technology and finding an exoplanet in the closer galaxy -- Andromeda -- could be considered science fiction. If the Milky Way has \textit{only} around $2\,\times10^{11}$ stars, how could be the Solar System unique? How could be the Earth the only planet with life on it? Until today, it is still difficult to answer those questions. The Solar System presents a logic structure where the small, rocky planets are closer to the Sun, and the gas giants are located beyond. But no other exoplanetary system have been found having similar features and even not exoplanetary life has been detected yet\footnote{By the date of writing this thesis.}. Our knowledge about the Solar System challenges the existence of other kinds of planetary architectures with close-in gas giants, for instance, suggesting that there is not a single, universal planetary process formation. Answering such questions makes studying exoplanets a relevant matter nowadays.

The increasing number of discovered exoplanets have been reached thanks to two techniques: Radial Velocity and Transits (see Figure \ref{exoplanets}). Each one of these techniques had their own advantages depending on the physical properties of the planetary system. When the orbit of the exoplanet is aligned with the line of sight from Earth, the pass of the planet in front of its host star causes that the star's flux decreases, proportionally to the size of the planet. Thus, its radius can be determined directly using the Transit method. In the other hand, the gravity due to the presence of a planet will set the center of mass of the system, in a place different from the star's center. Therefore, the star will move in its own small orbit with a size proportional to the mass of the planet. In this case, the Radial Velocity method unveils the planet's minimum mass ($M_{p}\sin i$). The perfect scenario comes when both methods can be used in the same planetary system, allowing to derive essential properties such as the real mass and the mean density of the planet, among others.

Anyway, the rising research field of exoplanets does not end with their discovery. Performing follow-up and proper characterization of discovered exoplanets is essential to obtain a full list of orbital and planetary parameters. For instance,  39\%  of the total number of discovered exoplanets have a proper measurement of their masses, and only 17\% have their mass and radius computed. Moreover, through follow-up studies several phenomena, such as their atmospheres, can be unveiled. 

\begin{figure}[t]
\centering
\includegraphics[width=1.0\columnwidth]{imagenes/exoplanets_hd.pdf}
\caption{Mass and orbital period distribution of Exoplanets, and their corresponding discovery methods in color. This plot only shows the exoplanets with measured masses. The Transit and Radial Velocity methods (in orange and blue, respectively) are the most successful techniques to discover extrasolar planets. However, both techniques are still more sensitive to detect Jupiter-mass planets rather than Earth-mass, but the Transit method is more sensitive to short-period planets.}
\label{exoplanets}
\end{figure}

Photometric follow-up of transiting exoplanets provides not only fundamental information about the exoplanet but their environment. Variations in the transit timing -- when the planet passes in front of its star -- could be induced by companions gravitationally bounded, such as other planets in their systems or even exomoons \citep{Kipping2009a,Kipping2009b}. As most of the space mission have strict fields of view, performing photometric follow-up with ground-based telescopes is crucial to reveal important features of their systems. The Transit Timing Variation (TTV) technique \citep{Holman2005,Agol2005} is an efficient and powerful tool to detect small, unseen companions of already discovered exoplanets (see Figure \ref{exoplanets}). Kepler provided the firsts breakthrough discoveries of exoplanets showing TTV \citep{Holman2010}, and after the mission ended some projects started aiming to complete its legacy searching for TTVs. The Transit Monitoring in the South project (TraMoS) is one of its kind in the southern hemisphere, aiming to detect multi-exoplanetary systems trough the TTV method.  


\section{This work}

This thesis is an extension of the Transit Monitoring in the South project (TraMoS, see Chapter \ref{chap:tramos}) started in 2008 by Sergio Hoyer and Patricio Rojo. I selected a new sample of transiting exoplanets, all of them hot Jupiters without any companions detected to date. I analyzed 27 new light curves of the hot Jupiters WASP-18b, WASP-19b, and WASP-77Ab searching for periodic variations in the transit time (TTV) that could suggest the presence of additional bodies in their systems. 

This thesis is organized as follows: Chapter \ref{chap:tramos} summarizes the TraMoS project, its goals, techniques and achievements to date. Also, this chapter includes what we can learn from transit exoplanets' light curves and Transit Timing Variations. Chapter \ref{chap:paper} shows the analysis and results of the hot Jupiters WASP-18b, WASP-19b and WASP-77Ab, including  their observations and data reduction, light curve and RV analysis, measurements of TTVs and upper mass limit for possible companions in their systems. Then, Chapter \ref{chap:tess}  recapitulates the current stage of the TraMoS project on which TESS data is being incorporated. Here, I re-analyze the TTV curves of the previous targets WASP-18b, WASP-19b, and WASP-77Ab with new light curves from TESS. Finally, Chapter \ref{chap:conclusion} contains a summary of this thesis as well as some ideas for future work. The Appendix section includes a brief description of the aperture photometry pipeline utilized, all the Python scripts used to generate each light curve, and an example of the PDF file obtained after the aperture photometry procedure.

