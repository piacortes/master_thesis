\chapter{Introduction}\label{chap:intro}
The idea of the existence of planets orbiting other stars different from our Sun has lived in the human imagination for centuries. During the sixteenth century, the Italian philosopher Giordano Bruno suggested\footnote{In the book: \textit{De l'infinito, universo e mondi}, 1584}, for the first time in history, that more planets could be outside the Solar System. More than four hundred years after Bruno's statement, the first exoplanet was confirmed in 1992. Surprisingly, it was not just one exoplanet but three, orbiting the pulsar PSR B1257+12 more than 1000 light-years away from us. Today, more than 4,000 exoplanets are confirmed, and thousands more are waiting for their confirmation.

%Even though the first claims of exoplanet detection began at the begin of the twentieth century\footnote{The van Maanen 2 star, which later was confirmed as a white dwarf.}, it was not until more than four hundred years after Bruno's statement, that  

Having billions of billions of galaxies in the Universe each one with millions of stars, the current number of discovered exoplanets is still very low. Furthermore, the distance at which we can detect exoplanets is limited by our current technology, thus finding a candidate in the closer galaxy -- Andromeda -- could be considered science fiction. If the Milky Way has \textit{only} around $2\,\times10^{11}$ stars, how could be the Solar System unique? How could the Earth be the only planet with life on it? Until today, it is still difficult to answer those questions. The Solar System presents a logic structure where the small, rocky planets are closer to the Sun, and the gas giants are located beyond. But no other exoplanetary system has been found having similar features and even not exoplanetary life has been detected yet. Our knowledge about the Solar System challenges the existence of other kinds of planetary architectures with close-in gas giants, for instance, suggesting that there is not a single, universal planetary process formation. Answering such questions makes studying exoplanets a relevant matter nowadays.

The discovery of three exoplanets orbiting the pulsar PSR B1257+12 is still very rare. The technique used to find them is called Pulsar Timing and it consists in measuring the variation of a pulsar's pulsation period due to the presence of a body orbiting it. As pulsars are not very common and we know that planet may be orbiting main-sequence stars -- as the Earth and all the Solar System --, they become the principal target on which the hunting of exoplanet is dedicated. Planets are small when compare to their host star, that is why most of the current discovery techniques are focused on observing the effects in the stars, or how what we observed from them changed, caused by the presence of planets. 

The most used and successful planet detection techniques are: transits, radial velocity, direct imaging, microlensing and astrometry. Direct imaging is the only technique in which the planet is observed directly but the star needs to be masked in order to only received the thermal emission from the planet. Microlensing is different from all the other discovery techniques because a background star is required to obtain the desired effect: gravitational lensing produced by the star with a planet (multiple lenses). As the primary lens is small -- around $1\,M_{\odot}$ -- the two images produced by the gravitational lensing effect are very close to each other and cannot be resolved with current imaging techniques. The presence of a planet in orbit around the lens star will cause a deviation in the magnification curve of the event. On the other hand, astrometry measurements can be used to discover exoplanets too. With the current high-precision in the position of the stars, even their displacement in the sky can be detected. Similar to the radial velocity technique, through astrometry the displacement of the star due to the gravity of a planet is measured. 

The two techniques that have contributed the most to the discovery of extrasolar planets are radial velocity and transits (see Figure \ref{exoplanets}). Each one of these techniques had their own advantages depending on the physical properties of the planetary system. When the orbit of the exoplanet is aligned with the line of sight from Earth, the pass of the planet in front of its host star causes that the star's flux decreases, proportionally to the square of the ratio between the radius of the planet and the star. Thus, the radius ratio between the planet and the star, $R_{p}/R_{*}$ can be determined directly using the transit method. On the other hand, the gravity due to the presence of a planet will set the center of mass of the system, in a place different from the star's center. Therefore, the star will move in its own small orbit with a radius proportional to the mass of the planet. In this case, the radial velocity method unveils the planet's minimum mass $M_{p}\sin i$. 

Anyway, the rising research field of exoplanets does not end with their discovery. Performing follow-up and proper characterization of discovered exoplanets are essential to obtain a full list of orbital and planetary parameters. For instance,  39\%  of the total number of discovered exoplanets have a proper measurement of their masses, and only 17\% have their mass and radius computed\footnote{Information obtained from the NASA Exoplanet Archive Database \url{https://exoplanetarchive.ipac.caltech.edu}}. 

%Moreover, through follow-up studies important information about the planet can be unveiled, even their atmospheric composition and dynamics.

One of the perfect scenarios comes when the two methods, transits and radial velocity, can be used in the same planetary system, allowing to derive essential properties of the planet and its orbit. Combining transits and radial velocity follow-up the whole system can be characterized. From transits we get the orbital inclination, then the real mass can be computed with radial velocity measurements. From transits we get the radius ratio of the system, assuming that the star is previously well characterized, the radius of the planet is derived. Then, as we already have the mass of the planet, the mean density can be estimated. This parameter gives a clue about the possible composition of the planet. Moreover, if the follow-up is performed combining transits with big telescopes, important information about the planet can be unveiled, perhaps even their atmospheric composition and dynamics.

\begin{figure}[t]
\centering
\includegraphics[width=1.0\columnwidth]{imagenes/exoplanets_hd.pdf}
\caption{Mass and orbital period distribution of Exoplanets, and their corresponding discovery methods in color. This plot only shows the exoplanets with measured masses. The transit and radial velocity methods (in orange and blue, respectively) are the most successful techniques to discover extrasolar planets. However, both techniques are still more sensitive to detect Jupiter-mass planets rather than Earth-mass, but the Transit method is more sensitive to short-period planets.}
\label{exoplanets}
\end{figure}

% Photometric follow-up of transiting exoplanets provides not only fundamental information about the exoplanet but their environment.

%The Transit method is today the most successful technique to discover extrasolar planets (see Figure \ref{amount_exoplanet}). The Kepler mission was launched in 2009 and during its almost nine years of operation provided thousands of discoveries due to this method.

Continued photometric monitoring not only increased the precision of planetary parameters but provides potential information about the planet and its environment. Through the observation of transit events during a long observing baseline, variability in the planetary system can be detected. For instance, the first measurement of two planets gravitationally bounded was made by \citep{Holman2010}, this discovery was made thanks to the continued observations performed by the Kepler mission. This phenomenon is detectable by observing the same transit event multiple times over an extended number of epochs. Any variability in the mid-transit times could be attributed to an additional body. Moreover, variability in transiting exoplanets -- specifically in the time of the mid-transit -- can be attributed not only to planetary companions, but exomoons\citep{Kipping2009a,Kipping2009b}, tidal orbital decay\citep{Hoyer2016,Hoyer2016b}, secular orbits changes\citep{Adams2006} or relativistic effects \citep{Heyl2007}, among others.

The Transit Monitoring in the South project (TraMoS) aims to detect multi-exoplanetary systems trough the variability in parameters of transiting exoplanets. To achieve this goal, extensive photometric follow-up of transit is being carried on, using one-meter class telescopes located in Chile.

% Variations in the transit timing -- when the planet passes in front of its star -- could be induced by companions gravitationally bounded, such as other planets in their systems or even exomoons \citep{Kipping2009a,Kipping2009b}. As most of the space mission have strict fields of view, performing photometric follow-up with ground-based telescopes is crucial to reveal important features of their systems. The Transit Timing Variation (TTV) technique \citep{Holman2005,Agol2005} is an efficient and powerful tool to detect small, unseen companions of already discovered exoplanets (see Figure \ref{exoplanets}). Kepler provided the firsts breakthrough discoveries of exoplanets showing TTV \citep{Holman2010}, and after the mission ended some projects started aiming to complete its legacy searching for TTVs. The Transit Monitoring in the South project (TraMoS) is one of its kind in the southern hemisphere, aiming to detect multi-exoplanetary systems trough the TTV method.  


\section{This work}

This thesis is an extension of the Transit Monitoring in the South project (TraMoS, see Chapter \ref{chap:tramos}) started in 2008 by Sergio Hoyer and Patricio Rojo. I selected a new sample of transiting exoplanets, all of them hot Jupiters without any companions detected to date. I analyzed 27 new light curves of the hot Jupiters WASP-18b, WASP-19b, and WASP-77Ab searching for periodic variations in the transit time (TTV) that could suggest the presence of additional bodies in their systems. 

This thesis is organized as follows: Chapter \ref{chap:tramos} summarizes the TraMoS project, its scientific goals, techniques, and achievements to date. Also, this chapter describes what we can learn from transit exoplanets' light curves and Transit Timing Variations. Chapter \ref{chap:paper} shows the analysis and results of the hot Jupiters WASP-18b, WASP-19b, and WASP-77Ab, including their observations and data reduction, light curve and RV analysis, measurements of TTVs and an upper mass limit for possible companions in their systems. Then, Chapter \ref{chap:tess}  recapitulates the current stage of the TraMoS project on which TESS data is being incorporated. Here, I re-analyze the TTV curves of the previous targets WASP-18b, WASP-19b, and WASP-77Ab with new light curves from TESS. Finally, Chapter \ref{chap:conclusion} contains a summary of this thesis as well as some ideas for future work. The Appendix section includes a brief description of the aperture photometry pipeline utilized, all the Python scripts used to generate each light curve, and an example of the PDF file obtained as a result of the aperture photometry procedure.

