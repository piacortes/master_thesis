\chapter{Introduction}\label{chap:intro}
The existence of planets orbiting other stars different from our Sun, lived in the human imagination for centuries. During the sixteenth century, the Italian philosopher Giordano Bruno suggested, for the first time in history, that more planet could be outside the Solar System. Even though the first claims of exoplanet detection began at the end of the nineteenth century, it was not until more than four hundred years after the statement of Bruno, that the first exoplanet was confirmed in 1992. Surprisingly, it was not just one exoplanet but three, orbiting the pulsar PSR B1257+12 more than 1000 light-years away from us. Today, more than 4,000 exoplanets are confirmed, and thousands more are waiting for their confirmation.

The increasing number of discovered exoplanets have been reached thank to two techniques: Radial Velocity and Transits. Each one of these techniques had their own advantages depending on the physical properties of the planetary system. When the orbit of the exoplanet is aligned with the line of sight from Earth, the pass of the planet in front of its host star causes that the star's flux decreases proportionally to the size of the planet. Thus, its radius, in comparison with the radius of the star, can be determined directly using the Transit method. In the other hand, the gravity due to the presence of a planet will set the center of mass of the system, in a place different from the star's center. Therefore, the star will move in its own small orbit with a size proportional to the mass of the planet. In this case, the Radial Velocity method unveils the planet's minimum mass ($M_{p}\sin i$). The perfect scenario comes when both methods can be used in the same planetary system, allowing to derive essential properties such as the real mass and the mean density of the planet.

\begin{figure}[H]
\centering
\includegraphics[width=1.0\columnwidth]{imagenes/exoplanets_hd.pdf}
\caption{Mass and orbital period distribution of Exoplanets, and their corresponding discovery methods in color. This plot only shows the exoplanets with known masses}
\label{exoplanets}
\end{figure}

\section{This work}

This thesis is an extension of the Transit Monitoring in the South project (TraMoS, see Chapter \ref{chap:tramos}) started in 2008 by Sergio Hoyer and Patricio Rojo. I selected a new sample of transiting exoplanets, all of them hot Jupiters without any companions detected to date.  I analized 22 new light curves of the hot Jupiters WASP-18b, WASP-19b, and WASP-77Ab searching for periodic variations in the transit time (TTV) that could suggest the presence of additional bodies in their systems. 

This thesis is organized as follows: Chapter \ref{chap:tramos} summarizes the TraMoS project, its goals, techniques and achievements to date. Also, this chapter includes what we can learn from transit exoplanets' light curves and Transit Timing Variations. Chapter \ref{chap:paper} shows the analysis and results of the hot Jupiters WASP-18b, WASP-19b and WASP-77Ab, including  their observations and data reduction, light curve and RV analysis, measurements of TTVs and upper mass limit for possible companions in their systems. Then, Chapter \ref{chap:tess}  recapitulates the current stage of the TraMoS project on which TESS data is being incorporated. Here, I re-analyze the TTV curves of the previous targets WASP-18b, WASP-19b, and WASP-77Ab with new light curves from TESS and also, preliminary results of WASP-36b are shown. Finally, Chapter \ref{chap:conclusion} contains the principal conclusions of this thesis as well as some ideas for future work. 


