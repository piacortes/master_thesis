\chapter{The Transit Monitoring in the South project}\label{chap:tramos}

High-precision long-term transit follow-ups provide tremendous opportunities in improving our understanding of exoplanets, leading to obtain more accurate measurements of planetary radius, especially those detected with ground-based transit surveys (e.g., HATNet and HATSouth, \citealt{Bakos2012}; SuperWASP, \citealt{Pollacco2006}; KELT, \citealt{Pepper2007}; TRES, \citealt{Alonso2007}, CSTAR, \citealt{WangS2014}). With improved photometry, we can refine planetary orbital ephemeris \citep{TEMP1}, which is vital to schedule future transit-related observations, such as Rossiter-Mclaughlin effect measurement \citep{Nutzman2011,Sanchis2011,Sanchis2013,WangS2018} and transmission spectrum follow-up \citep{Mancini2016b,Mackebrandt2017}.

%Long-term photometric follow-up also provides a unique chance to study the variations of the orbital periods. A recent study shows the apparent orbital decay in the WASP-12 system \citep{Patra2017}, which intrigues a series of theoretical studies \citep{Millholland2018,Weinberg2017} to discuss the potential mechanisms. The transit follow-up also plays an important role in exoplanet system which shows interesting Transit Timing Variations (TTV) \citep{Ballard2011,Ford2012a,Steffen2012,Fabrycky2012,Mancini2016,WangS2017,Wu2018}. 

%\cite{Ballard2018} predicted that around 5\% of planets discovered by TESS \citep{Ricker2014} will show TTVs. Transit follow-up of these targets is very critical, because most of them will only be monitored for $\sim27$ days, whereas the typical TTV period is around years. 

%Furthermore, extended TTV studies are crucial to confirm or rule out exoplanetary systems, in cases where space-based observations will not cover the long-time scales required to characterize them \citep{vonEssen2018}. Thus, combining ground and space-based observations will be crucial. 

%The TTV method \citep{Miralda2002,Agol2005,Holman2005}  also provides a powerful tool to detect additional low-mass planets in hot Jupiter systems, which is usually hard to find by using other techniques \citep{Steffen2012b}. Many efforts have been devoted to this field \citep{Pal2011,Hoyer2012,Hoyer2013,Szabo2013}, but so far only two hot Jupiters have been found to accompanies with additional close-in planets (WASP-47: \cite{Becker2015}, and Kepler-730: \cite{Canas2019}). The accurate occurrence rate of the `WASP-47-like' system is still unknown.

%To refine orbital parameters of currently known exoplanets, and to search for additional planets by using TTV method, we organized the Transit Monitoring in the South hemisphere (TraMoS) project \citep{Hoyer2011} since 2008. We uses one-meter class telescopes in the north of Chile to conduct high-precision long-term transit follow-up. 

%Following the previous efforts from the TraMoS project, in this work, we present new light curves of three hot Jupiters: WASP-18b, WASP-19b, WASP-77Ab. Combining our new light curves, and archival photometric and radial velocity data sets, we refined the orbital and physical parameters of the systems, and constrained the upper mass limit of potential additional planetary companions. 