\chapter{Summary and future work}\label{chap:conclusion}

The TraMos project started more than 10 years ago aiming to refine orbital and physical parameters of already known exoplanets, and to search for variation in the transit time -- or other planet parameters -- that could suggest the existence of smaller bodies on their systems. Since the beginning, the targets have been hot Jupiters due to their short-periods, allowing to obtain more consecutive transit light curves in a reduced timespan. 

%To date, we have not detected periodic TTV signal on any of the analyzed exoplanets.

The Transit Timing Variation (TTV) technique was proposed as a technique to detect Earth-mass planet in gas giant systems, due to their mutual dynamical interaction. The variation in the transit time is enhanced when the two bodies are near to Mean Motion Resonances (MMR), therefore, this kind of interaction is easier to detect from ground-based follow-up. Through the years, the TTV technique has demonstrated to be an efficient and powerful tool not only to detect small exoplanets in the Earth-mass regime but to obtain essential information of multi-exoplanetary systems. By analyzing and modeling TTV curves of planetary systems in near MMR, the bodies' masses can be estimated, whence RV could be no longer a requirement for masses estimation.

During this Master project, I studied three exoplanets: WASP-18b, WASP-19b, and WASP-77Ab. For these targets, I analyzed a total of twentysix new light curves: eight of WASP-18b, nine of WASP-19b and nine of WASP-77Ab. In the case of WASP-77Ab, four light curves were taken from the Exoplanet Transit Database (ETD)\footnote{\url{var2.astro.cz/ETD}} to obtain a wider time coverage. For the light curves coming from the TraMoS project, I started from performing aperture photometry of the targets using our data reduction pipeline (for further details see \ref{chap:pipeline}), in order to obtain high-precision light curves. Then, I used EXOFASTv2 \cite{Eastman2013,Eastman2017} to get the best-fit for the light curve's data and hence, to refine their orbital and physical parameters and obtain their transit times.

The TTV curves were constructed from the residuals between the observed transit time and the expected from the linear ephemeris. None of the targets studied in this thesis, shows a clear, periodic sinusoidal signal in the variation of their transit times and moreover, the structure of the transit times are comparable with a random distribution after a Lomb-Scargle analysis. This suggests the lack of close-in companions in MMR in their systems. However, upper mass limit of hypothetical perturbers -- placed not only in near MMR -- could be computed with the time-scatter from the linear ephemeris. The hot Jupiter WASP-18b shows 83 seconds of scatter, WASP-19b shows 75 seconds and WASP-77ab shows 121 seconds. The latter being the target with larger deviation from the linear ephemeris, but at the same time, the target with less transit times.

In summary, the upper mass limit for possible perturbers that could produced the observed scatter in the transit time are:
\begin{itemize}
\item WASP-18b: $1.9\,M_{\odot}$, $4.0\,M_{\odot}$, $5.0\,M_{\odot}$, $25\,M_{\odot}$, $10\,M_{\odot}$, $7.0\,M_{\odot}$, $50\,M_{\odot}$ in 1:3, 2:5, 1:2, 2:1, 5:2, 3:1 and 4:1 resonances, respectively. 
\item WASP-19b: $2.0\,M_{\odot}$, $6.0\,M_{\odot}$, $0.9\,M_{\odot}$, $3.0\,M_{\odot}$, $2.0\,M_{\odot}$ in 1:2, 5:3, 2:1, 5:2 and 3:1 resonances, respectively. 
\item WASP-77Ab: $1.0\,M_{\odot}$, $3.5\,M_{\odot}$, $2.0\,M_{\odot}$, $7.0\,M_{\odot}$, $6.0\,M_{\odot}$, $10\,M_{\odot}$, $3.5\,M_{\odot}$, $4.5\,M_{\odot}$ in 1:3, 1:2, 3:5, 2:3, 5:3, 7:4, 2:1 and 3:1 resonances, respectively.
\end{itemize}

After the release of TESS data, I included transit times of the targets studied in this thesis, in order to extend their TTV curves and confirm the lack of periodic variations in the transit time. The results came out as expected and the scatter in the transit time were reduced for the three targets: 47 seconds for WASP-18b, 65 seconds for WASP-19b and 86 seconds for WASP-77Ab. Moreover, I refined each orbital period obtaining a precision of the order of milli-seconds. 

To date, none of the previous and current targets of the TraMoS project have shown TTVs. As they are all hot Jupiters, these studies support the theory of "alone hot Jupiters", on which this kind of exoplanets are probably alone in their systems or accompanied of small bodies not in MMR. How WASP-47b and Kepler-730b, the only two hot Jupiters showing TTVs, have close-in companions is still unknown. The migration theory of hot Jupiters could give us a possible explanation of how these gas giant ended-up with companions in MMR, as they are supposed to start their migration process in far-away orbits. While migrating, they could catch other bodies to then settling down in close orbits. But the lack of more hot Jupiters with TTVs suggest that probably during migration they cleaned-up their orbits of small bodies. 

The TraMoS catalogue stores more than 400 transits events of near 140 exoplanets. As future work I propose to continue performing ground-based photometric follow-up of the targets in this catalogue, as well as new targets, combining the efforts with TESS data of Sector 1 to 13. In this way, we could obtain better estimations of TTVs with wider time-span data from TraMoS, and an important amount of consecutive transit times from TESS. In the near future, I will analyze the exoplanets WASP-36b and HATS-34b, which each one has five transit times from TraMoS and twelve and twenty six from TESS, respectively. For WASP-22, WASP-46, WASP-45, WASP-23 and WASP-123, TESS provided five, seventeen, eight, seven and thirteen light curves, respectively. These targets are within the 30 targets with most light curves from TraMoS, therefore, they should be the next in line.

The reduction pipeline is fully functional and provides high-precision photometry, nevertheless several improvements were left behind because of the lack of time to implement them. For example, the coordinates selection could be done inside the program, to avoid mistakes when typing them by hand before calling the pipeline. Any new idea to improvement the reduction pipeline could be done within an undergrad project or graduate semester-workshop.
 