\chapter{Summary and future work}\label{chap:conclusion}

The TraMos project started more than 10 years ago aiming, on one hand, to refine orbital and physical of already known exoplanets, and on the other, to search for variation in the transit time -- or other planet parameters -- that could suggest the existence of smaller bodies on their systems. Since the beginning, the targets has been hot Jupiter due to their short-periods allowing to obtain more consecutive transit light curves in a reduced timespan. 

%To date, we have not detected periodic TTV signal on any of the analyzed exoplanets.

The Transit Timing Variation (TTV) technique was proposed as a technique to detect Earth-mass planet in gas giant systems, due to their mutual dynamical interaction. The variation in the transit time is enhanced when the two bodies are near to Mean Motion Resonances (MMR), therefore, this kind of interaction is easier to detect from ground-based follow-up. Through the years, the TTV technique has demonstrated to be an efficient and powerful tool not only to detect small exoplanets in the Earth-mass regime but to obtain essential information of multi-exoplanetary systems. By analyzing and modeling TTV curves of planetary systems in near MMR, the bodies' masses can be estimated, whence RV could be no longer a requirement for masses estimation.

During this Master project, I studied three exoplanets: WASP-18b, WASP-19b, and WASP-77Ab. For these targets, I analyzed a total of 26 light curves: 8 of WASP-18b, 9 of WASP-19b and 9 of WASP-77Ab. In the case of WASP-77Ab, 4 light curves were taken from the Exoplanet Transit Database (ETD)\footnote{\url{var2.astro.cz/ETD}} to obtain a wider time coverage. For the light curves coming from the TraMoS project, I started from performing aperture photometry of the targets using our data reduction pipeline (for further details see \ref{chap:pipeline}), in order to obtain high-precision light curves. Then, I used EXOFASTv2 \cite{Eastman2013,Eastman2017} to get best-fit for the light curve's data and hence, to refine their orbital and physical parameters and obtain their transit times.

The TTV curves were constructed from the residuals between the observed transit time and the expected from the linear ephemeris. Any of the targets studied in this thesis, shows a clear, periodic sinusoidal signal. This suggest the lack of close-in companions in MMR in their systems. However, upper mass limit of hypothetical perturbers -- not only in MMR -- could be computed with the time-scatter from the linear ephemeris. The hot Jupiter WASP-18b shows 83 seconds of scatter, WASP-19b shows 75 seconds and WASP-77ab shows 121 seconds, being the target with larger deviation from the linear ephemeris, but at the same time, the target with less transit times.

