\chapter{Analysis and Results of WASP-18b, WASP-19b, and WASP-77Ab}\label{chap:paper}

\section*{TraMoS V: Updated ephemeris and transit timing variations of the hot Jupiters WASP-18b, WASP-19b and WASP-77Ab\footnote{Copy of paper submitted to A\%A on July 9, 2019.}}

\chapterauthor{P\'ia Cort\'es-Zuleta, Patricio Rojo, Songhu Wang, Tobias C. Hinse, Sergio Hpyer, Bastian Sanhueza, Patricio Correa-Amaro, Julio Albornoz}

\subsection*{Abstract}

We present 22 new transit observations for the exoplanets WASP-18b, WASP-19b, and WASP-77Ab, from the Transit Monitoring in the South (\emph{TraMoS}) project. We simultaneously model our newly collected transit light curves, as well as archival photometry and radial velocity data to obtain refined physical and orbital parameters. We did not find significant $\rm TTV_{RMS}$ variations larger than 83, 75, and 121 seconds for WASP-18b, WASP-19b, and WASP-77Ab, respectively. Dynamical simulations were carried out to constrain the masses of a possible perturber. The observed RMS could be produced by a perturber body with an upper limit mass of 5 and $7~M_{\oplus}$ in 1:2 and 3:1 resonance, and $25~M_{\oplus}$ in 2:1 resonance for WASP-18b. In the case of WASP-19b, companions with masses up to 0.9, 2 and $6~M_{\oplus}$, in 2:1, 3:1, and 5:3 resonances, reproduce the RMS. In this system, we also discard a hypothetical perturber of $500~M_{\oplus}$ in 1:3 resonance, because of the mass limit from RV variations. For WASP-77Ab a planet with masses between $1-7~M_{\oplus}$ in 1:2, 1:3, 2:1, 2:3, 3:1, 3:5, 5:3 resonances could reproduce the observed RMS in the TTVs. Finally, using a Lomb-Scargle period search we find no evidence of a periodic trend on our TTV data for the three exoplanets.

\section{Introduction}

High-precision long-term transit follow-ups provide tremendous opportunities in improving our understanding of exoplanets, leading to obtain more accurate measurements of planetary radius, especially those detected with ground-based transit surveys (e.g., HATNet and HATSouth, \citealt{Bakos2012}; SuperWASP, \citealt{Pollacco2006}; KELT, \citealt{Pepper2007}; TRES, \citealt{Alonso2007}, CSTAR, \citealt{WangS2014}). With improved photometry, we can refine planetary orbital ephemeris \citep{TEMP1}, which is vital to schedule future transit-related observations, such as Rossiter-Mclaughlin effect measurement \citep{Nutzman2011,Sanchis2011,Sanchis2013,WangS2018} and transmission spectrum follow-up \citep{Mancini2016b,Mackebrandt2017}.

Long-term photometric follow-up also provides a unique chance to study the variations of the orbital periods. A recent study shows the apparent orbital decay in the WASP-12 system \citep{Patra2017}, which intrigues a series of theoretical studies \citep{Millholland2018,Weinberg2017} to discuss the potential mechanisms. The transit follow-up also plays an important role in exoplanet system which shows interesting Transit Timing Variations (TTV) \citep{Ballard2011,Ford2012a,Steffen2012,Fabrycky2012,Mancini2016,WangS2017,Wu2018}. 

\cite{Ballard2018} predicted that around 5\% of planets discovered by TESS \citep{Ricker2014} will show TTVs. Transit follow-up of these targets is very critical, because most of them will only be monitored for $\sim27$ days, whereas the typical TTV period is around years. 

Furthermore, extended TTV studies are crucial to confirm or rule out exoplanetary systems, in cases where space-based observations will not cover the long-time scales required to characterize them \citep{vonEssen2018}. Thus, combining ground and space-based observations will be crucial. 

The TTV method \citep{Miralda2002,Agol2005,Holman2005}  also provides a powerful tool to detect additional low-mass planets in hot Jupiter systems, which is usually hard to find by using other techniques \citep{Steffen2012b}. Many efforts have been devoted to this field \citep{Pal2011,Hoyer2012,Hoyer2013,Szabo2013}, but so far only two hot Jupiters have been found to accompanies with additional close-in planets (WASP-47: \cite{Becker2015}, and Kepler-730: \cite{Canas2019}). The accurate occurrence rate of the `WASP-47-like' system is still unknown.

To refine orbital parameters of currently known exoplanets, and to search for additional planets by using TTV method, we organized the Transit Monitoring in the South hemisphere (TraMoS) project \citep{Hoyer2011} since 2008. We uses one-meter class telescopes in the north of Chile to conduct high-precision long-term transit follow-up. 

Following the previous efforts from the TraMoS project, in this work, we present new light curves of three hot Jupiters: WASP-18b, WASP-19b, WASP-77Ab. Combining our new light curves, and archival photometric and radial velocity data sets, we refined the orbital and physical parameters of the systems, and constrained the upper mass limit of potential additional planetary companions. 

%\subsection{Systems}
WASP-18b is a transiting hot Jupiter discovered by \citet{Hellier2009} within the WASP-South transit survey \citep{Pollacco2006}. It is an extremely close-in planet orbiting a F6 type star with a period of 0.94 days. Regarding its physical properties, WASP-18b is about ten times more massive than Jupiter with approximately the same size ($M_{P}=10.3\,{\rm M_{Jupiter}}$, $R_{P}=1.1\,{\rm R_{Jupiter}}$). Even though a rapid orbital decay was predicted theoretically \citep{Hellier2009}, it is not observed yet \citep{Wilkins2017} and new theoretical models proposed a variation of less than 4 seconds in the transit time over a 20-yr baseline \citep{CollierCameron2018}.

The hot Jupiter WASP-19b was first reported by \cite{Hebb2010}. It is known as one of the currently hot Jupiters with shortest orbital period ($P=0.788\,{\rm days}$). With a mass of $1.15\,{\rm M_{Jupiter}}$ and a radius of $1.31\,{\rm R_{Jupiter}}$, the planet orbits an active G8 dwarf.

The third exoplanet we followed-up in this work, WASP-77Ab, was first presented by \cite{Maxted2013}. WASP-77Ab has a mass of $1.8\,{\rm M_{Jupiter}}$ and a radius of $1.2\,{\rm R_{Jupiter}}$, and orbital period of 1.36 days. It transits a G8 star in the visual binary system with a separation of 3.3 arcsec.

This paper is organized as follows. In Section~\ref{obs} are summarized the photometric observations and their reduction process. In Section~\ref{lc} we present the new light curves of the targets and the description of the technique used to obtain their orbital and physical parameters. The principal results and their consequences are presented in Section~\ref{res}. Finally, a summary and conclusions are described in Section~\ref{summary}.


\section{Observations and Data Reduction}

We collected 8 light curves for WASP-18b between 2009 and 2017, 9 light curves for WASP-19b between 2011 and 2017, and 5 light curves for WASP-77Ab between 2015 and 2017. We included 4 transits of WASP-77Ab from the Exoplanet Transit Database (ETD) in order to cover a larger timespan.

All of the photometry are collected by using either the Danish 1.54 m telescope at ESO La Silla Observatory, or the SMARTS 1 m at Cerro Tololo Observatory (CTIO), except for one transit of WASP-77Ab that was observed with the Warsaw 1.3 m at Las Campanas Observatory (LCO). The log of our observations is shown in Table \ref{log_table}. All the new light curves used for this work are presented in Figure~\ref{transits}.

For the photometric observations conducted on the Danish telescope, we used the Danish Faint Object Spectrograph and Camera (DFOSC) instrument, which has a $2{\rm K} \times 2{\rm K}$ CCD with a 13.7 x 13.7 arcmin$^2$ field of view (FoV) and a pixel scale of 0.39" per pixel. To reduce the readout time, some of the Danish 1.54 m images were windowed to only include the target star and its closest reference stars. The observations of the transits of WASP-18b during 2016 and 2017 were forced to be windowed due to a malfunction of the CCD. For those transits, only one reference star was used to perform the photometry.

The SMARTS 1 m has the Y4KCam instrument which is a $4{\rm K} \times 4{\rm K}$ CCD camera with a $20\times20$ arcmin$^2$ FoV and a pixel scale of 0.289" per pixel. 

For the observation on the Warsaw 1.3 m telescope, we used a $2048 \times 4096$ CCD camera chip with a 1.4 square degrees of FoV and 0.26" per pixel scale. No windowing or binning was used during the observations on both SMARTS 1m and the Warsaw 1.3m telescope.

As suggested by \cite{Southworth2009}, most of our observation, especially those conducted after 2011, used the defocus technique, which allows longer exposure times in bright targets and improves the photometric precision. We adjust the exposure time during the observations if the weather is not ideal. The recorded Julian Date in the Coordinated Universal Time (${\rm JD_{UTC}}$) were converted into Barycentric Julian Date in the Barycentric Dynamical Time standard (${\rm BJD_{TDB}}$) by following the procedure as in \citet{Eastman2010}.

We reduced the data by using our custom pipeline. It follows the standard procedures of reduction, calibration, and aperture photometry, but customized for each used instrument. The pipeline semi-automatically finds the best aperture and ring size, for the sky that produces the light curve with less RMS. Then, we manually choose the reference stars to produce the differential light curves for each targets.


\begin{table*}
\caption{Log of Observations}             
\label{log_table}      
\centering          
\begin{tabular}{cccccccc}
\hline\hline       
Target & Date & Epoch\footnotemark{a} & Telescope & Filter & N & $t_{exp}$\footnotemark{b} & RMS\footnotemark{c} \\
& (UTC) &       &           &       &                   & (seconds) & (mmag)\\
\hline  
WASP-18 & 2009 Oct 28 &-1904 & SMARTS 1 m & $I$ & 1412 & 1.5 & 8.49  \\
 				& 2009 Oct 29 & -1903 & SMARTS 1 m & $I$ & 1435 & 2 & 5.67 \\
				& 2009 Oct 30 & -1902 & SMARTS 1 m & $I$ & 1198 & 2  & 4.50 \\
 				& 2011 Sep 06 & -1184 & SMARTS 1 m & $I$ & 203 & 15 & 2.40 \\
				& 2016 Sep 24\footnotemark{d} & 776 & Danish 1.54 m & $I$ & 138 & 90 & 1.05  \\
				& 2016 Sep 25\footnotemark{d} & 777 &Danish 1.54 m & $I$ & 159 & 90  & 0.96  \\
			    & 2016 Sep 26\footnotemark{d} & 778 & Danish 1.54 m & $I$ & 113 & 90 & 0.87  \\ \smallskip
				& 2017 Sep 29\footnotemark{d} & 1169 & Danish 1.54 m & $R$ & 330 & 30 & 2.53 \\
WASP-19 & 2011 Apr 22 & -923 & SMARTS 1 m &   $I$ &  626  & 12  & 4.31   \\
     & 2011 Dec 24 & -611 & SMARTS 1 m &   $I$ & 364  & 18  & 35.9 \\
     & 2013 Mar 13 & -47 & Danish 1.54 m & $R$ & 336 & 35 & 2.15\\
     & 2013 Apr 20 & 1 & Danish 1.54 m & $R$ & 153 & 100 & 0.80 \\
     & 2015 Mar 04 & 867 & Danish 1.54 m & $R$ & 235 & 60 & 0.84\\
     & 2016 Apr 14 & 1383 & Danish 1.54 m & $I$ & 87 & 100 & 0.71\\
     & 2017 Feb 14 & 1771 & Danish 1.54 m & $I$ & 137 & 90 & 0.79\\
     & 2017 Apr 08 & 1838 & Danish 1.54 m & $R$ & 125 & 90 & 0.81\\\smallskip
     & 2017 Oct 03 & 2064 & Danish 1.54 m & $R$ & 43 & 110 & 1.70\\ 
 WASP-77 & 2013 Aug 20 & -659 & ETD\footnotemark{e} & $clear$ & 103 & 120 & 3.87  \\
    & 2013 Oct 30 & -606 & ETD\footnotemark{e} & $clear$ & 690 & 12 & 5.91 \\
    & 2015 Sep 29 & -92 & Danish 1.54 m & $R$ &  244 & 30 &  0.84\\
      & 2015 Oct 03 & -89 & Danish 1.54 m & $R$ &  138 & 60 &  1.84\\
      & 2016 Sep 26 & 175 & Danish 1.54 m & $I$ &  90 & 90 &  0.47\\
      & 2016 Sep 30 & 177 & ETD\footnotemark{e}& $clear$ & 66 & 180 & 2.74 \\
      & 2016 Oct 07 & 183 & Warsaw 1.3 m & $I$ & 237 & 60 &  2.38 \\ 
      & 2016 Dec 09 & 229 & ETD\footnotemark{e} & $R$ & 57 & 180 & 2.11 \\ 
      & 2017 Oct 01 & 447 & Danish 1.54 m & $B$ &  224 & 30 &   3.48\\
\hline                  
\end{tabular}
\footnotetext{a}{The epoch 0 is $T_{0}$ in Tables , for WASP-18b, WASP-19b and WASP-77Ab, respectively.}
\footnotetext{b}{For the variable exposure times, we consider the average during the night.}
\footnotetext{c}{The RMS values were computed from the best fitted model of each light curve.}
\footnotetext{d}{Light curves obtained with only one reference star.}
\end{table*}

\begin{table}

\caption{Example photometry of WASP-18, WASP-19 and WASP-77A}
\label{examplephot}
\centering
\begin{tabular}{c c c c}
\hline \hline
Target & ${\rm BJD_{TDB}}$\footnotemark{a} & Relative flux & Error\\
\hline
    WASP-18b &  2457658.658241   & 1.00168 & 0.00078 \\
             &  2457658.660591   & 1.00138 & 0.00080 \\
             &  2457658.661771   & 1.00195 & 0.00082 \\
             &  2457658.662940   & 1.00261 & 0.00085 \\
             &  2457658.664109   & 1.00137 & 0.00086 \\\smallskip
      & ...           & ...         &    ... \\
    WASP-19b & 2457086.543926   & 1.00099 & 0.00086  \\
             & 2457086.544916   & 1.00173 & 0.00091  \\
             & 2457086.545905   & 1.00139 & 0.00086  \\
             & 2457086.546895   & 1.00045 & 0.00094  \\
             & 2457086.547886   & 1.00064 & 0.00093  \\\smallskip
     &  ...           & ...         &    ...  \\
WASP-77Ab & 2457299.78624 & 1.00229 & 0.00028 \\ 
         & 2457299.78764 & 1.00116 & 0.00022 \\
         & 2457299.78855 & 1.00201 & 0.00022 \\
         & 2457299.78946 & 1.00216 & 0.00022 \\
         & 2457299.79092 & 1.00133 & 0.00021 \\
         & ...           & ...         &    ... \\    
\hline
\end{tabular}
\footnotetext{These tables are available in machine-readable form.\\
\footnotetext{a}{The column time was converted to (${\rm BJD_{TDB}}$), following the procedure of \cite{Eastman2010}.}
}
\end{table}

\section{Light curve and RV analysis}

To obtain the refined orbital and physical parameters of WASP-18b, WASP-19b, and WASP-77Ab, as well as their transit mid-time ($T_{c}$), we used EXOFASTv2 \citep{Eastman2013,Eastman2017} to model the light curves together with archived RV data \cite{Hellier2009, Hebb2010, Maxted2013}.

EXOFASTv2 is an IDL code designed to simultaneously fit transits and radial velocity measurements obtained from different filters or different telescopes. It uses the Differential Evolution Markov chain Monte Carlo (DE-MCMC) method to derive the values and their uncertainties of the stellar, orbital and physical parameters of the system. 

The stellar parameters of WASP-18, WASP-19, and WASP-77A were computed using the MESA Isochrones and Stellar Tracks (MIST) model \citep{Dotter2016} included in EXOFASTv2. We applied Gaussian priors in surface gravity $\log{g}$, effective temperature $T_{\rm eff}$, and metallicity [Fe/H] of the stars, from \cite{Hellier2009},\cite{Hebb2010} and \cite{Maxted2013} for WASP-18, WASP-19 and WASP-77A, respectively. 
We were not able to separate the contribution of the two companions of the binary system WASP-77, because the separation is 3.3 arcsec, but our photometry aperture is about 10 arcsec. Thus, we computed the dilution factor -- fraction of the light that comes from the companion star -- for each filter of our data set in order to get the real transit depth of WASP-77Ab. Because of the lack of good quality magnitude measurements for the fainter companion WASP-77B in the $B$, $I$, $R$ and $clear$ pass bands, we derived them from the \emph{Gaia} magnitude ($G=11.8356$) assuming Black Body radiation. The derived magnitudes for WASP-77B are $V=11.97$, $B=12.72$, $R=11.57$, $I=10.95$ and $\emph{clear}=11.78$.

We set previous published values as uniform priors for the DE-MCMC in all the transit, RV parameters, quadratic limb darkening coefficients and $T_{c}$. The priors were taken from the discovery papers of WASP-18b \citep{Hellier2009}, WASP-19b \citep{Hebb2010} and WASP-77Ab \citep{Maxted2013}. 

In order to reduce significantly the convergence time of the chains during the EXOFASTv2 fitting, we started from shorter chains. Thus, the total time to complete that run is reduced. After it finished, we took the values from its best model and used them as priors for the next short run. This process was repeated until the chains were converged and well-mixed.

The best-fitted model is presented in Figure~\ref{transits} for our transit data from the TraMoS project, and in Figure~\ref{rv} for the RV archival data.

\begin{figure*}
\centering
\includegraphics[width=1.0\textwidth]{imagenes/light_curves2.pdf}
\caption{Light curves of WASP-18, WASP19 and WASP77 during 8, 9 and 9 different transits, respectively, from the TraMoS project. The fitted best model from EXOFASTv2 is shown as a light blue solid line for WASP-18b, orange for WASP-19b and pink for WASP-77Ab. To the right of each panel are the corresponding residuals of the model. For clarity, both light curves and their residual are offset artificially. The epoch number is indicated above each light curve. The technical information about each observation is listed in Table~\ref{log_table}.}
\label{transits}
\end{figure*}

\begin{figure*}
%\vspace{0.1cm}\hspace{0.2cm}
\includegraphics[width=1.0\textwidth]{imagenes/rv_all.pdf}
\caption{Radial velocity observations of WASP-18, WASP-19 and WASP-77A from \cite{Hellier2009}, \cite{Hebb2010} and \cite{Maxted2013}, respectively. The best fitted model from the joint modeling of RV and light curves with EXOFASTv2 is in solid line color: light blue for WASP-18b, orange for WASP-19b and pink for WASP-77Ab. The residuals of the model are shown at the bottom panel of each figure.}
\label{rv}
\end{figure*}

\begin{figure}
%\vspace{0.2cm}\hspace{0.2cm}
\centering
\includegraphics[width=0.7\columnwidth]{imagenes/phase.pdf}
\caption{Phased light curv of WASP-18b, WASP-19b and WASP-77Ab transits, from the TraMoS project. The three data set of light curves are fitted simultaneously with RV archival data using EXOFASTv2, in order to estimate the orbital and physical parameters of the system. In the top panel, the light blue solid line is the best fitting model for WASP-18b, and bellow are the residuals in color grey. The same for WASP-19b in color orange at the center panel, and for WASP-77Ab in color pink at the bottom panel.}
\label{phase}
\end{figure}

\section{Results and Discussion}

\subsection{Transit Parameters and Physical Properties}\label{transitparams}

\subsubsection{WASP-18b}

The resulting parameters from the global fit of WASP-18 in comparison the with results of the discovery paper \cite{Hellier2009} and the most recent analysis with TESS data \citep{Shporer2018}, are listed in Table~\ref{wasp18}. While in \cite{Hellier2009} the analysis was performed combining photometry and RV data, in \cite{Shporer2018} only photometric data was used. 

%Stellar params
As the stellar spectroscopic priors were taken from the discovery paper \cite{Hellier2009}, our results for the stellar mass $M_*$ and radius $R_*$ are in good agreement with theirs, as expected, as well as the rest of the stellar parameters. \cite{Shporer2018} do not present results of stellar parameters.

%Primary transit params
In the case of the primary transit parameters, the greatest difference is found in the radius of the planet in stellar radii $R_{p}/R_{*}$. Our reported $R_{p}/R_{*}$ is $7.8\sigma$ and $4.4\sigma$ larger that the reported by \cite{Hellier2009} on the discovery paper and the recent result from \cite{Shporer2018}, respectively.  Our transit duration $T_{14}$ is also $3.4\sigma$ larger than the value from \cite{Hellier2009}.

% RV params
For the radial velocity parameters, the RV semi-amplitude derived from our analysis is consistent with the value of \cite{Hellier2009}, as the same data was used. 

% Derived params
Finally, the derived parameters of the system are, in general, in good agreement with the values from \cite{Hellier2009} and \cite{Shporer2018}. 

\begin{landscape}
\begin{longtable}{llccc}
\caption{System parameter of WASP-18}
\label{wasp18}
\centering
\tabularnewline
\hline 
Parameter & Units & This work & \cite{Hellier2009} & \cite{Shporer2018} \\
\hline
\smallskip\\\multicolumn{2}{l}{Stellar Parameters:}&\smallskip\\
~~~~$M_*$  &Mass (\(M_\odot\))  &$1.294^{+0.063}_{-0.061}$ & $1.25\pm0.13$ &  \\
~~~~$R_*$  &Radius (\(R_\odot\))  &$1.319^{+0.061}_{-0.062}$ &$1.26^{+0.067}_{-0.054}$ & \\
~~~~$L_*$  &Luminosity (\(L_\odot\))  &$2.68^{+0.28}_{-0.26}$ & &\\
~~~~$\rho_*$  &Density (cgs)  &$0.795^{+0.11}_{-0.089}$ & $ 0.707^{+0.056}_{-0.096}$&\\
~~~~$\log{g}$  & Surface gravity (cgs)  &$4.310^{+0.036}_{-0.033}$ &$4.367^{+0.028}_{-0.042}$ & \\
~~~~$T_{\rm eff}$  &Effective Temperature (K)  &$6432\pm48$ & $6400\pm100$& \\
~~~~$[{\rm Fe/H}]$  &Metallicity   &$0.107\pm0.080$ & $0.00\pm0.09$ & \\
~~~~$Age$  &Age (Gyr)  &$1.57^{+1.4}_{-0.94}$ & $0.5-1.5$ & \\
%~~~~$d$  &Distance (pc)  &$122.2^{+9.9}_{-8.2}$ & $100\pm10$ & \\

\smallskip\\\multicolumn{2}{l}{Planetary Parameters:}&\smallskip\\
~~~~$R_P$  &Radius (\rj)  &$1.310\pm0.071$ & $1.106^{+0.072}_{-0.054}$& $1.192\pm0.038$\\
~~~~$M_P$  &Mass (\mj)  &$10.48^{+0.42}_{-0.40}$ & $10.30\pm0.69$ & \\
~~~~$P$  &Period (days)  &$0.94145236\pm(49)$ & $0.94145299\pm(87)$& $0.9414576^{(+34)}_{(-35)}$ \\
~~~~$e$  &Eccentricity   &$0.0061^{+0.0089}_{-0.0044}$ & &\\
~~~~$a$  &Semi-major axis (AU)  &$0.02054^{+0.00033}_{-0.00032}$ & $0.02045\pm0.00067$&\\
~~~~$\omega_*$  &Argument of Periastron (Degrees)  &$-100^{+110}_{-120}$ &  &\\
~~~~$\rho_P$  &Density (cgs)  &$5.79^{+0.97}_{-0.78}$& $7.73^{+0.78}_{-1.27}$\footnotemark{b} & \\
~~~~$logg_P$  &Surface gravity   &$4.180^{+0.044}_{-0.041}$ & $4.289^{+0.027}_{-0.050}$ &\\
~~~~$T_{eq}$  &Equilibrium temperature (K)  &$2485^{+53}_{-56}$ & $2384^{+58}_{-30}$ & \\
~~~~$\Theta$  &Safronov Number   &$0.254^{+0.015}_{-0.014}$ & &\\
~~~~$\fave$  &Incident Flux (\fluxcgs)  &$8.66^{+0.77}_{-0.75}$ & & \\

\smallskip\\\multicolumn{2}{l}{Primary Transit Parameters:}&\smallskip\\
~~~~$T_0$  &Transit time (\bjdtdb)  &$2456740.80560\pm(19)$ & $2454221.48163\pm(38)$ & $2458361.048072^{(+34)}_{(-35)}$\\
~~~~$i$  &Inclination (Degrees)  &$83.5^{+2.0}_{-1.6}$ & $86.0\pm2.5$ & $84.31^{+0.40}_{-0.37}$\\
~~~~$R_P/R_*$  &Radius of planet in stellar radii   &$0.1021\pm0.0011$ &  $0.0935\pm0.0011$ & $0.09721^{+0.00016}_{-0.00017}$\\
~~~~$a/R_*$  &Semi-major axis in stellar radii   &$3.35^{+0.15}_{-0.13}$ & & $3.523^{+0.028}_{-0.027}$\\
~~~~$b$  &Impact parameter   &$0.433^{+0.07}_{-0.10}$ & $0.25\pm0.15$ & $0.349^{+0.020}_{-0.022}$\\
~~~~$\delta$  &Transit depth (fraction)  &$0.01041\pm0.00022$ & & $0.009449^{+0.000032}_{-0.000032}$\\
~~~~$u_{1,I}$  &linear LD coeff., I band  &$0.207\pm0.019$ & &\\
~~~~$u_{2,I}$  &quadratic LD coeff., I band  &$0.313\pm0.019$& &\\
~~~~$u_{1,R}$  &linear LD coeff., R band  & $0.257\pm0.045$ & &\\
~~~~$u_{2,R}$  &quadratic LD coeff., R band  &$0.309\pm0.048$ & &\\
~~~~$T_{14}$  &Total transit duration (days)  &$0.0931^{+0.0011}_{-0.0010}$ & $0.08932\pm0.00068$ &\\
~~~~$P_T$  &A priori non-grazing transit prob   &$0.268^{+0.011}_{-0.012}$ & &\\
~~~~$P_{T,G}$  &A priori transit prob   &$0.328^{+0.014}_{-0.015}$ & &\\
~~~~$\tau$  &Ingress/egress transit duration (days)  &$0.0107\pm0.0010$ & &\\

\smallskip\\\multicolumn{2}{l}{RV Parameters:}&\smallskip\\
~~~~$ecos{\omega_*}$  &   &$-0.0004^{+0.0038}_{-0.0045}$ & &\\
~~~~$esin{\omega_*}$  &   &$-0.0008^{+0.0056}_{-0.0092}$ & &\\
~~~~$K$  &RV semi-amplitude (m/s)  &$1807^{+34}_{-36}$ & $1818.3\pm8.0$ &\\
~~~~$M_P\sin i$  &Minimum mass (\mj)  &$10.40\pm0.40$ & &\\
\smallskip\\\multicolumn{2}{l}{Secondary Eclipse Parameters:}&\smallskip\\
~~~~$T_S$  &Time of eclipse (\bjdtdb)  &$2457657.3076^{+0.0023}_{-0.0027}$ & &\\
~~~~$b_S$  &Eclipse impact parameter   &$0.431^{+0.070}_{-0.100}$ & &\\
~~~~$\tau_S$  &Ingress/egress eclipse duration (days)  &$0.0106^{+0.0011}_{-0.0010}$ & &\\
~~~~$T_{S,14}$  &Total eclipse duration (days)  &$0.0929\pm0.0017$ & &\\
~~~~$P_S$  &A priori non-grazing eclipse prob   &$0.269^{+0.010}_{-0.011}$ & &\\
~~~~$P_{S,G}$  &A priori eclipse prob   &$0.330^{+0.013}_{-0.014}$ & &\\
\hline 
\end{longtable}
\end{landscape}
%\tablefoot{
%\tablefoottext{a}{Value converted to cgs units multiplying by the Sun density $\rho_{\odot}=1.408\,$cgs.}
%\tablefoottext{b}{Value converted to cgs units multiplying by the Jupiter density $\rho_{J}=1.33\,$cgs.}
%\tablefoottext{c}{Values enclosed in parentheses correspond to the uncertainties of the last digits of the nominal value.}}




\subsubsection{WASP-19b}
The results of the global fit of WASP-19 are listed in Table~\ref{tab:wasp19}, in comparison with the previous values from the discovery paper \citep{Hebb2010}, and a more recent work \citep{Lendl2013}. 

To estimate the stellar parameters of WASP-19, we used as priors the stellar spectroscopic parameters from \cite{Hebb2010}. Thus, in general, our results are in agreement with the those from the discovery paper. The most important discrepancies are the density of the star $\rho_*$ and the surface gravity $\log{g}$, showing $+2.5\sigma$ and $-3.2\sigma$ difference, respectively. Comparing with the results from \cite{Lendl2013}, ours are all in good agreement.

For values of the primary transit parameters obtained from the light curves, the greatest differences are found in the orbital inclination $i$ and the total transit duration $T_{14}$. We report an inclination value $5.1\sigma$ smaller than \cite{Hebb2010}, but in agreement with the estimate of \cite{Lendl2013}. In the other hand, our estimation of $T_{14}$ is significantly larger than \cite{Hebb2010} by $9\sigma$, but the difference is only $3.5\sigma$ when compared with \cite{Lendl2013}. We also report a preciser impact parameter $b$ and transit depth $\delta$.

As the same RV data set from the discovery paper \citep{Hebb2010} was used to perform our analysis, the almost identical values in the RV semi-amplitude $K$ is not a surprise. Moreover, the values from \cite{Lendl2013} are also in agreement. 

The planetary parameters derived from the light curve and radial velocity analysis are almost all in good agreement with the comparison works. The only parameter with a difference greater than $3\sigma$ is our estimation of the Equilibrium Temperature $T_{eq}$ compared with the result of \cite{Hebb2010}. However, our result is in better agreement with \cite{Lendl2013} by less than $2\sigma$.

\begin{landscape}
\begin{longtable}{llccc}
\caption{System parameter of WASP-19}
\label{tab:wasp19}
\centering
\tabularnewline
\hline 
~~~Parameter & Units & This work & \cite{Hebb2010}\footnote{a} & \cite{Lendl2013}\\
\hline
\smallskip\\\multicolumn{2}{l}{Stellar Parameters:}&\smallskip\\
~~~~$M_*$\dotfill &Mass (\(M_\odot\))\dotfill &$0.965^{+0.091}_{-0.095}$ & $0.95\pm0.10$ & $0.968^{+0.084}_{-0.079}$ \\
~~~~$R_*$\dotfill &Radius (\(R_\odot\))\dotfill &$1.006^{+0.031}_{-0.034}$ & $0.93^{+0.05}_{-0.04}$& $0.994\pm0.031$\\
~~~~$L_*$\dotfill &Luminosity (\(L_\odot\))\dotfill &$0.905^{+0.071}_{-0.069}$\\
~~~~$\rho_*$\dotfill &Density (cgs)\dotfill &$1.339^{+0.056}_{-0.058}$ & $1.19^{+0.12}_{-0.11}$\footnote{b}  & $1.384^{+0.055}_{-0.051}$\footnote{b}\\
~~~~$\log{g}$\dotfill &Surface gravity (cgs)\dotfill &$4.417^{+0.020}_{-0.021}$ & $4.48\pm0.03$\\
~~~~$T_{\rm eff}$\dotfill &Effective Temperature (K)\dotfill &$5616^{+66}_{-65}$ & $5500\pm100$\\
~~~~$[{\rm Fe/H}]$\dotfill &Metallicity \dotfill &$0.04^{+0.25}_{-0.30}$ & $0.02\pm0.09$\\
~~~~$Age$\dotfill &Age (Gyr)\dotfill &$6.4^{+4.1}_{-3.5}$ & $5.5^{+9.0}_{-4.5}$\\
%~~~~$d$\dotfill &Distance (pc)\dotfill &$270.2\pm1.7$\\

\smallskip\\\multicolumn{2}{l}{Planetary Parameters:}&\smallskip\\
~~~~$R_P$\dotfill &Radius (\rj)\dotfill &$1.415^{+0.044}_{-0.048}$ & $1.28\pm0.07$ & $1.376\pm0.046$\\
~~~~$M_P$\dotfill &Mass (\mj)\dotfill &$1.154^{+0.078}_{-0.080}$ & $1.14\pm0.07$ & $1.165\pm0.068$\\
~~~~$P$\dotfill &Period (days)\dotfill &$0.78883852^{+(75)}_{-(82)}$ & $0.7888399\pm(8)$  & $0.7888390\pm(2)$\\
~~~~$e$\dotfill &Eccentricity \dotfill &$0.0126^{+0.014}_{-0.0089}$ & & $0.0077^{+0.0068}_{-0.0032}$\\
~~~~$a$\dotfill &Semi-major axis (AU)\dotfill &$0.01652^{+0.00050}_{-0.00056}$ & $0.0164^{+0.0005}_{-0.0006}$ & $0.01653\pm0.00046$\\
~~~~$\omega_*$\dotfill &Argument of Periastron (Degrees)\dotfill &$51^{+89}_{-190}$ & $-76^{+112}_{-23}$ & $43^{+28}_{-67}$\\
~~~~$\rho_P$\dotfill &Density (cgs)\dotfill &$0.506^{+0.031}_{-0.030}$ & $0.54^{+0.07}_{-0.06}$& $0.595^{+0.036}_{-0.033}$\footnote{c}\\
~~~~$logg_P$\dotfill &Surface gravity \dotfill &$3.155^{+0.018}_{-0.019}$ & $3.20\pm0.03$ & $3.184\pm0.015$\\
~~~~$T_{eq}$\dotfill &Equilibrium temperature (K)\dotfill &$2113\pm29$ & $1993^{+32}_{-33}$ & $2058\pm40$\\
~~~~$\Theta$\dotfill &Safronov Number \dotfill &$0.0279^{+0.0012}_{-0.0011}$\\
~~~~$\fave$\dotfill &Incident Flux (\fluxcgs)\dotfill &$4.52^{+0.26}_{-0.24}$\\

\smallskip\\\multicolumn{2}{l}{Primary Transit Parameters:}&\smallskip\\
~~~~$T_0$\dotfill & Transit Time (\bjdtdb)\dotfill &$2456402.7128^{+(17)}_{-(14)}$ & $2454775.3372\pm(2)$ & $2456029.59204\pm(13)$\\
~~~~$i$\dotfill &Inclination (Degrees)\dotfill &$79.08^{+0.34}_{-0.37}$ & $80.8\pm0.8$  & $79.54\pm0.33$\\
~~~~$R_P/R_*$\dotfill &Radius of planet in stellar radii \dotfill &$0.14410^{+0.00049}_{-0.00050}$ & $0.1425\pm0.0014$\\
~~~~$a/R_*$\dotfill &Semi-major axis in stellar radii \dotfill &$3.533^{+0.048}_{-0.052}$ &  & $3.573\pm0.046$\\
~~~~$b$\dotfill &Impact parameter \dotfill &$0.6671^{+0.0087}_{-0.0091}$ & $0.62\pm0.03$ & $0.645\pm0.012$\\
~~~~$\delta$\dotfill &Transit depth (fraction)\dotfill &$0.02077\pm0.00014$ & $0.0203\pm0.0004$ & $0.02018\pm0.00021$\\
~~~~$u_{1,I}$\dotfill &linear LD coeff., I band\dotfill &$0.287^{+0.027}_{-0.029}$&\\
~~~~$u_{2,I}$\dotfill &quadratic LD coeff., I band\dotfill &$0.263\pm0.024$&\\
~~~~$u_{1,R}$\dotfill &linear LD coeff., R band \dotfill &$0.383^{+0.029}_{-0.032}$\\
~~~~$u_{2,R}$\dotfill &quadratic LD coeff., R band\dotfill &$0.246^{+0.027}_{-0.025}$\\
~~~~$T_{14}$\dotfill &Total transit duration (days)\dotfill &$0.06697^{+0.00031}_{-0.00030}$ & $0.0643^{+0.0006}_{-0.0007}$ & $0.06586^{+0.00033}_{-0.00031}$\\
~~~~$P_T$\dotfill &A priori non-grazing transit prob \dotfill &$0.2426^{+0.0066}_{-0.0051}$\\
~~~~$P_{T,G}$\dotfill &A priori transit prob \dotfill &$0.3246^{+0.0089}_{-0.0069}$\\
~~~~$\tau$\dotfill &Ingress/egress transit duration (days)\dotfill &$0.01459\pm0.00035$\\

\smallskip\\\multicolumn{2}{l}{RV Parameters:}&\smallskip\\
~~~~$e\cos{\omega_*}$\dotfill & \dotfill &$-0.0027^{+0.0077}_{-0.013}$ & $0.004\pm0.009$ &$0.0024\pm0.0020$ \\
~~~~$e\sin{\omega_*}$\dotfill & \dotfill &$0.0016^{+0.014}_{-0.0092}$ & $-0.02\pm0.02$ & $0.000\pm0.005$\\
~~~~$K$\dotfill &RV semi-amplitude (m/s)\dotfill &$255.4^{+6.1}_{-6.2}$ & $256\pm5$  & $257.7\pm2.9$\\
~~~~$M_P\sin i$\dotfill &Minimum mass (\mj)\dotfill &$1.133^{+0.078}_{-0.079}$\\

\smallskip\\\multicolumn{2}{l}{Secondary Eclipse Parameters:}&\smallskip\\
~~~~$T_S$\dotfill &Time of eclipse (\bjdtdb)\dotfill &$2455169.3621^{+(41)}_{-(51)}$ & $2456030.77766\pm(88)$\\
~~~~$b_S$\dotfill &Eclipse impact parameter \dotfill &$0.670^{+0.020}_{-0.017}$ & $0.652\pm0.015$\\
~~~~$\tau_S$\dotfill &Ingress/egress eclipse duration (days)\dotfill &$0.01472^{+0.00085}_{-0.00066}$\\
~~~~$T_{S,14}$\dotfill &Total eclipse duration (days)\dotfill &$0.06812^{+0.00087}_{-0.00074}$\\
~~~~$P_S$\dotfill &A priori non-grazing eclipse prob \dotfill &$0.2415\pm0.0021$\\
~~~~$P_{S,G}$\dotfill &A priori eclipse prob \dotfill &$0.3232\pm0.0030$\\
\hline
%\end{tabular}
%\tablefoot{
%\tablefoottext{a}{For comparison, the results from \cite{Hellier2009} that considered free eccentricity were used.}
%%\tablefoottext{b}{Values converted to cgs units multiplying by the Sun density $\rho_{\odot}=1.408\,$cgs.}
%\tablefoottext{c}{Values converted to cgs units multiplying by the Jupiter density $\rho_{J}=1.33\,$cgs}
%\tablefoottext{d}{Values enclosed in parentheses correspond to the uncertainties of the last digits of the nominal value.}}
\end{longtable}
\end{landscape}

\subsubsection{WASP-77Ab}
Table~\ref{tab:wasp77} lists the results of the global fit of WASP-77Ab, in comparison with the values from its discovery paper \citep{Maxted2013} on which photometry and RV data were used. No other previous work has reported bulk measurements for this system.

Almost all the stellar parameters are in agreement with \citep{Maxted2013}, except for a $-9.7\sigma$ difference in the stellar surface gravity $\log{g}$, where our reported value is preciser.  

The primary transit parameters, as well as the RV parameters and the derived planetary parameters, are consistent with the results from \cite{Maxted2013}.

\begin{landscape}
\begin{longtable}{llcc}
\caption{System parameter of WASP-77A}
\label{tab:wasp77}
\centering
\tabularnewline
\hline 
~~~~~Parameter & Units & This work & \cite{Maxted2013}\\
\hline
\smallskip\\\multicolumn{2}{l}{Stellar Parameters:}&\smallskip\\
~~~~$M_*$\dotfill &Mass (\(M_\odot\))\dotfill &$0.903^{+0.066}_{-0.059}$ & $1.002\pm0.045$\\
~~~~$R_*$\dotfill &Radius (\(R_\odot\))\dotfill &$0.910^{+0.025}_{-0.023}$ & $0.955\pm0.015$\\
~~~~$L_*$\dotfill &Luminosity (\(L_\odot\))\dotfill &$0.743^{+0.065}_{-0.058}$ & \\
~~~~$\rho_*$\dotfill &Density (cgs)\dotfill &$1.692^{+0.056}_{-0.069}$\footnote{a} & $1.629^{+0.023}_{-0.028}$\footnote{a}\\
~~~~$\log{g}$\dotfill &Surface gravity (cgs)\dotfill &$4.476^{+0.014}_{-0.015}$ & $4.33\pm0.08$\\
~~~~$T_{\rm eff}$\dotfill &Effective Temperature (K)\dotfill &$5617\pm72$ & $5500\pm80$\\
~~~~$[{\rm Fe/H}]$\dotfill &Metallicity \dotfill &$-0.10^{+0.10}_{-0.11}$ & $0.00\pm0.11$\\
~~~~$Age$\dotfill &Age (Gyr)\dotfill &$6.2^{+4.0}_{-3.5}$ & $0.5-1.0$\\

\smallskip\\\multicolumn{2}{l}{Planetary Parameters:}&\smallskip\\
~~~~$R_P$\dotfill &Radius (\rj)\dotfill &$1.183^{+0.034}_{-0.031}$ & $1.21\pm0.02$\\
~~~~$M_P$\dotfill &Mass (\mj)\dotfill &$1.650^{+0.082}_{-0.075}$ & $1.76\pm0.06$\\
~~~~$P$\dotfill &Period (days)\dotfill &$1.3600290^{+(18)}_{-(20)}$ & $1.3600309\pm(20)$\\
~~~~$e$\dotfill &Eccentricity \dotfill&$0.0074^{+0.0075}_{-0.0051}$\\
~~~~$a$\dotfill &Semi-major axis (AU)\dotfill &$0.02323^{+0.00056}_{-0.00052}$ & $0.0240\pm0.00036$\\
~~~~$\omega_*$\dotfill &Argument of Periastron (Degrees)\dotfill &$-30^{+170}_{-120}$\\
~~~~$\rho_P$\dotfill &Density (cgs)\dotfill &$1.240^{+0.060}_{-0.067}$ & $1.33\pm0.04$\footnote{b}\\
~~~~$logg_P$\dotfill &Surface gravity \dotfill &$3.467^{+0.012}_{-0.015}$ & $3.441\pm0.008$\\
~~~~$T_{eq}$\dotfill &Equilibrium temperature (K)\dotfill &$1695^{+25}_{-24}$\\
~~~~$\Theta$\dotfill &Safronov Number \dotfill &$0.0717\pm0.0021$\\
~~~~$\fave$\dotfill &Incident Flux (\fluxcgs)\dotfill &$1.87^{+0.11}_{-0.10}$\\

\smallskip\\\multicolumn{2}{l}{Primary Transit Parameters:}&\smallskip\\
~~~~$T_0$\dotfill &Transit Time (\bjdtdb)\dotfill &$2457420.88439^{(+80)}_{(-85)}$ & $2455870.44977\pm(20)$\\
~~~~$i$\dotfill &Inclination (Degrees)\dotfill &$88.91^{+0.74}_{-0.95}$ & $89.4^{+0.4}_{-0.7}$\\
~~~~$R_P/R_*$\dotfill &Radius of planet in stellar radii \dotfill &$0.13352^{+0.00074}_{-0.00070}$\\
~~~~$a/R_*$\dotfill &Semi-major axis in stellar radii \dotfill &$5.493^{+0.060}_{-0.075}$\\
~~~~$b$\dotfill &Impact parameter \dotfill &$0.105^{+0.089}_{-0.071}$ & $0.06^{+0.07}_{-0.05}$\\
~~~~$\delta$\dotfill &Transit depth (fraction)\dotfill &$0.01783^{+0.00020}_{-0.00019}$\\
~~~~$u_{1,B}$\dotfill &linear LD coeff., B band \dotfill &$0.680\pm0.054$&\\
~~~~$u_{2,B}$\dotfill &quadratic LD coeff., B band\dotfill &$0.140^{+0.052}_{-0.053}$&\\
~~~~$u_{1,clear}$\dotfill &linear LD coeff., \emph{clear} band \dotfill &$0.386\pm0.029$\\
~~~~$u_{2,clear}$\dotfill &quadratic LD coeff., \emph{clear} band \dotfill &$0.227\pm0.029$&\\
~~~~$u_{1,I}$\dotfill &linear LD coeff., I band \dotfill &$0.311\pm0.025$&\\
~~~~$u_{2,I}$\dotfill &quadratic LD coeff., I band \dotfill &$0.294\pm0.033$&\\
~~~~$u_{1,R}$\dotfill &linear LD coeff., R band \dotfill &$0.312\pm0.023$\\
~~~~$u_{2,R}$\dotfill &quadratic LD coeff., R band \dotfill &$0.237^{+0.029}_{-0.028}$\\
~~~~$T_{14}$\dotfill &Total transit duration (days)\dotfill &$0.08952^{+0.00053}_{-0.00051}$\\
~~~~$P_T$\dotfill &A priori non-grazing transit prob \dotfill &$0.1578^{+0.0029}_{-0.0025}$\\
~~~~$P_{T,G}$\dotfill &A priori transit prob \dotfill &$0.2064^{+0.0039}_{-0.0033}$\\
~~~~$\tau$\dotfill &Ingress/egress transit duration (days)\dotfill &$0.01075^{+0.00032}_{-0.00015}$\\

\smallskip\\\multicolumn{2}{l}{RV Parameters:}&\smallskip\\
~~~~$e\cos{\omega_*}$\dotfill & \dotfill &$-0.0033^{+0.0041}_{-0.0065}$\\
~~~~$e\sin{\omega_*}$\dotfill & \dotfill &$-0.0002^{+0.0061}_{-0.0073}$\\
~~~~$K$\dotfill &RV semi-amplitude (m/s)\dotfill &$323.4^{+3.7}_{-3.3}$ & $321.9\pm3.9$\\
~~~~$M_P\sin i$\dotfill &Minimum mass (\mj)\dotfill &$1.649^{+0.082}_{-0.075}$\\

\smallskip\\\multicolumn{2}{l}{Secondary Eclipse Parameters:}&\smallskip\\
~~~~$T_S$\dotfill &Time of eclipse (\bjdtdb)\dotfill &$2457659.5665^{+0.0038}_{-0.0056}$\\
~~~~$b_S$\dotfill &Eclipse impact parameter \dotfill &$0.104^{+0.089}_{-0.071}$\\
~~~~$\tau_S$\dotfill &Ingress/egress eclipse duration (days)\dotfill &$0.01076^{+0.00035}_{-0.00023}$\\
~~~~$T_{S,14}$\dotfill &Total eclipse duration (days)\dotfill &$0.0895^{+0.0013}_{-0.0014}$\\
~~~~$P_S$\dotfill &A priori non-grazing eclipse prob \dotfill &$0.1578^{+0.0018}_{-0.0012}$\\
~~~~$P_{S,G}$\dotfill &A priori eclipse prob \dotfill &$0.2063^{+0.0025}_{-0.0016}$\\
\hline
%\end{tabular}
%\tablefoot{
%\tablefoottext{a}{Value converted to cgs units multiplying by the Sun density $\rho_{\odot}=1.408\,$cgs.}
%\tablefoottext{b}{Value converted to cgs units multiplying by the Jupiter density $\rho_{J}=1.33\,$cgs.}
%\tablefoottext{c}{Values enclosed in parentheses correspond to the uncertainties of the last digits of the nominal value.}
%}
\end{longtable}
\end{landscape}

\subsection{Transit Timing Variations}\label{ttvsection}

 A transit timing variation is represented trough a difference in time from the transit mid-time $T_c$, for a planet in Keplerian motion following a linear ephemeris of the orbital period. The TTVs were computed considering our transit times from the TraMoS project, as well as including previous published ones. A refined orbital period was linearly fitted, considering 19, 59 and 11 transit times of WASP-18b, WASP-19b, and WASP-77Ab, respectively. Along the linear fit, we also tested a second degree polynomial fit to analyze a possible orbital decay. Both fit considered the errors of the data. In Figure~\ref{ttv} are presented all the TTV values for the transit times of WASP-18b, WASP-19b and WASP-77Ab.

If the planet stays in a Keplerian orbit, its transit time $T_{c}$ of each epoch $E$ should follow a linear function of the orbital period $P$.

\begin{equation} \label{eq1}
T_{\rm C}(E)=T_{\rm C}(0)+E \cdot P
\end{equation}

Where $T_{c}(0)$ is the optimal transit time in an arbitrary zero epoch. The best-fitted values of $T_{c}(0)$ for WASP-18b, WASP-19b and WASP-77Ab, are listed in Tables~\ref{wasp18},~\ref{tab:wasp19} and \ref{tab:wasp77}, respectively.


\subsubsection{WASP-18b}
For this system, the proposed linear ephemeris equation is:

\begin{equation} \label{eq1_w18}
T_{\rm C}(E)=2456926.27460\pm(94)+E \cdot 0.941452232\pm(89)
\end{equation}

The orbital period $P$ in Eq.~\ref{eq1_w18} is in complete agreement with the value computed only with the TraMoS light curves in Table~\ref{wasp18}.

Table~\ref{times_wasp18} lists the TTV of our transit times and also, of data from previous works \citep{Triaud2010,Hellier2009,Maxted2013b} of WASP-18b. 

The top panel of Figure~\ref{ttv} is the linear plot of TTV versus epoch for this planet. The deviations of the transit times from the linear ephemeris has an RMS of 83 seconds. The greater deviations come from the transit time of the epochs $-1904$ and $-1184$, which are over the linear ephemeris by $2.5\sigma$ and $1.9\sigma$, respectively. If those values are removed, the RMS decreases to 61 seconds. Considering all the transit times in Table~\ref{times_wasp18} without the epochs $-1904$ and $-1184$, all the TTVs lie within $1.6\sigma$ in the linear fit. 

The epoch $-1184$ has the greatest error in our sample because it is not a complete transit.

When testing the goodness of the linear fit, $\chi^{2}_{red} =0.56$, while for a second degree polynomial is $\chi^{2}_{red}=0.48$, therefore an orbital decay can be discarded in agreement with theoretical estimations \citep{CollierCameron2018}.  

\begin{table}[H]
%\tablewidth{0pt}
\centering
\caption{Transit mid-times for WASP-18b}
\label{times_wasp18}
\begin{tabular}{cccc}
\hline \hline
Epoch & Transit mid-time & TTV & References\\
      & (${\rm BJD_{TDB}}$) & (min) & \\
\hline 
-2873 & $2454221.48238$ & $0.1\pm1.5$ & 1\\
-2402 & $2454664.9061$ & $-0.4\pm1.4$ & 2 \\
-1904 & $2455133.7472$ & $-3.4\pm1.7$ & This work \\
-1903 & $2455134.6914$ & $0.6\pm1.7$ & This work \\
-1902 & $2455135.6331$  & $0.9\pm1.7$  & This work \\
-1811 & $2455221.3042$ & $-0.6\pm1.4$  & 3\\
-1629 & $2455392.6474$ & $-2.2\pm1.5$& 3 \\
-1601 & $2455419.0083$ & $-1.8\pm2.2$& 3\\
-1587 & $2455432.1897$ & $-0.3\pm1.4$ & 3\\
-1546 & $2455470.7885$ & $-1.4\pm1.4$ & 3\\
-1543 & $2455473.6144$ & $0.9\pm1.9$& 3\\
-1457 & $2455554.5786$ & $-0.2\pm1.5$&3 \\
-1440 & $2455570.5842$ & $1.2\pm1.6$& 3\\
-1184 & $2455811.5970$ &  $2.7\pm5.9$ & This work  \\
-1115 & $2455876.5559$ & $0.8\pm2.3$ & 3 \\ 
776 & $2457656.84078$ & $-1.1\pm1.4$ & This work  \\
777 & $2457657.78359$   & $0.9\pm1.4$ & This work\\
778 & $2457658.72404$ & $-0.6\pm1.4$ & This work \\
1169 & $2458026.83186$ & $-0.6\pm1.5$ & This work  \\
\hline
\end{tabular}
%\tablebib{(1)~\citet{Hellier2009};
%(2) \citet{Triaud2010}; (3) \citet{Maxted2013}.}
\end{table} 

\subsubsection{WASP-19b}
%The best values of the parameters in Equation \ref{eq1} for WASP-19b are $T_{c}=2456402.7128^{+(17)}_{-(14)}\,[\rm{BJD_{TDB}}]$ and $P=0.78883852^{+(75)}_{-(82)}\,{\rm days}$. 

The proposed equation for linear ephemeris, considering 59 transit times of WASP-19b is:
\begin{equation} \label{eq1_w19}
T_{\rm C}(E)=2456402.7128\pm(16)+E \cdot 0.788838858\pm(47)
\end{equation}

The TTV values from the proposed linear ephemeris are listed in Table \ref{times_wasp19}, including transit times from the previous works \citep{Hebb2010,Anderson2010,Lendl2013,Tregloan2013,Bean2013,Mancini2013}. 

At the middle panel of Figure~\ref{ttv} are the TTV values versus epoch, for all the transit time considered in this work.

The RMS from the linear ephemeris is about 75 seconds. The epoch of $-1011$ is the only one with a transit time deviation above $1.5\,\sigma$ from the linear ephemeris. If it is removed, then the RMS decreases to 69 seconds. Moreover, in our data the epoch $-611$ has one of the greatest errors due to bad weather conditions.

Considering all the transit times from Table~\ref{times_wasp19}, the linear fit has $\chi^{2}_{red} = 0.16$. This an indication of an overestimation of the errors. A second degree polynomial was also tested in order to reject or not a possible orbital decay. The goodness of that fit is $\chi^{2}_{red}=0.12$. 

\begin{longtable}{cccc}
\caption{Transit mid-times for WASP-19b}
\label{times_wasp19}
\tabularnewline
\hline \hline
Epoch & Transit mid-time & TTV & References\\
      & (${\rm BJD_{TDB}}$) & (min) &  \\
\hline
-2063 & $2454775.3372$ & $-1.6\pm3.2$ & 1\\
-2061 & $2454776.91566$ & $-0.5\pm2.3$ & 2 \\
-2010 & $2454817.14633$ & $-0.6\pm2.3$ & 3 \\
-1525 & $2455199.73343$ & $-0.2\pm2.6$& 4\\
-1459 & $2455251.79657$ & $-0.6\pm2.3$& 5 \\
-1458 & $2455252.58544$ & $-0.5\pm2.3$& 5 \\
-1454 & $2455255.74077$ & $-0.6\pm2.3$& 5 \\
-1449 & $2455259.68448$ & $-1.3\pm2.4$& 4 \\ 
-1431 & $2455273.88282$ & $-2.4\pm2.5$& 4 \\
-1399 & $2455299.12768$ & $0.6\pm2.4$& 4 \\
-1354 & $2455334.6254$ & $0.5\pm2.3$ & 6\\
-1349 & $2455338.56927$ & $0.1\pm2.3$& 3 \\ 
-1330 & $2455353.55659$ & $-0.8\pm2.3$& 6\\
-1317 & $2455363.81131$ & $-1.1\pm2.4$& 4  \\
-1311 & $2455368.54285$ & $-3.3\pm3.8$& 6 \\ 
-1094 & $2455539.72327$ & $0.2\pm2.4$ & 3 \\
-1056 & $2455569.69826$ & $-1.1\pm2.4$ & 3\\
-1039 & $2455583.10979$ & $0.8\pm2.6$ & 4 \\
-1037 & $2455584.68693$ & $0.0\pm2.3$ & 3 \\
-1025 & $2455594.15188$ &  $-1.6\pm3.3$ & 6 \\
-1016 & $2455601.25164$ & $-1.3\pm2.5$ & 6\\
-1014 & $2455602.83138$ & $1.7\pm2.4$ & 3 \\
-1011 & $2455605.19414$ & $-3.8\pm3.5$ & 6 \\
-1009 & $2455606.77464$ & $0.3\pm2.3$ & 3 \\
-1008 & $2455607.56241$ & $-1.2\pm2.4$ & 3 \\
-989 & $2455622.55057$ & $-0.9\pm2.3$ & 3 \\
-987 & $2455624.12787$ & $-1.5\pm3.1$ & 6 \\
-976 & $2455632.80612$ & $0.0\pm2.3$ & 3 \\
-947 & $2455655.68222$ & $-0.3\pm2.4$ & 3 \\
-928 & $2455670.66976$ & $-0.9\pm2.5$ & 3 \\
-923 & $2455674.61367$ & $-1.3\pm2.4$ & This work \\
-919 & $2455677.77038$ & $0.7\pm3.6$ & 6  \\
-905 & $2455688.81201$ & $-2.4\pm5.3$ & 6 \\
-904 & $2455689.60276$ & $0.4\pm2.4$ & 6 \\
-900 & $2455692.75674$ & $-1.6\pm4.3$ & 6 \\
-899 & $2455693.54639$ & $-0.5\pm2.3$ & 6  \\
-886 & $2455703.79933$ & $-3.3\pm6.4$ & 6 \\
-885 & $2455704.59078$ & $0.5\pm2.4$ & 6 \\
-880 & $2455708.534626$ & $0.0\pm 2.3$ & 6  \\
-654 & $2455886.81234$ &  $0.2\pm3.8$ & 6 \\
-642 & $2455896.27611$ & $-3.1\pm3.8$ & 6 \\
-618 & $2455915.20980$ & $-0.9\pm2.5$ & 6  \\
-613 & $2455919.15485$ & $0.4\pm 2.7$ & 6\\
-611 & $2455920.7353$ & $0.0\pm5.4$ & This work \\
-609 & $2455922.30966$ & $-0.4\pm8.3$ & 6 \\
-511 & $2455999.6163$ & $0.2\pm2.3$ & 7 \\
-483 & $2456021.70374$ & $0.1\pm2.3$ & 7 \\
-473 & $2456029.5925$ & $0.7\pm 2.4$ &3 \\
-468 & $2456033.53643$ &  $0.3\pm 2.3$ & 6  \\
-430 & $2456063.51174$ & $-0.5\pm 2.3$ & 3 \\
-86 & $2456334.87208$ & $-0.8\pm2.4$ & 4 \\
-47 & $2456365.6373$  & $-0.1\pm2.3$& This work \\
1 & $2456403.50158$ &  $-0.1\pm2.3$ & This work \\
729 & $2456977.77722$  & $1.3\pm2.3$ & 8 \\
867 & $2457086.63571$ & $-0.5\pm2.3$& This work \\
1383 & $2457493.67676$   & $-0.2\pm2.3$& This work  \\
1771 & $2457799.74612$ &  $-0.3\pm2.3$ & This work \\
1838 & $2457852.597807$ & $-0.7\pm2.3$ & This work\\ 
2064 & $2458030.8751$  & $-1.5\pm3.2$& This work \\
\hline
%\end{tabular}
%\tablebib{(1)~\citet{Hebb2010};
%(2) \citet{Anderson2010}; (3) \citet{Lendl2013}; (4) Exoplanet Transit Database (ETD) \url{var2.astro.cz/ETD}; (5) \citet{Tregloan2013}; (6) \citet{Mancini2013}; (7) \citet{Bean2013}; (8) \citet{Sedaghati2015}.}
\end{longtable} 

\subsubsection{WASP-77Ab}

As in the previous targets, we computed a refined linear ephemeris equation for WASP-77A considering 11 transit times:
\begin{equation} \label{eq1_w77}
T_{\rm C}(E)=2457420.88439\pm(85)+E \cdot 1.36002854\pm(52)
\end{equation}

In Table~\ref{tab:wasp77} are listed the TTV values of our transit times and from previous works \citep{Turner2016,Maxted2013} of WASP-77Ab and, at the bottom of Figure~\ref{ttv} are plotted versus epoch. The scatter of all the transit times is about $RMS=121$ seconds. 

The epochs $175$ and $229$ are above $2\,\sigma$ from the expected transit time following the linear ephemeris, while all the others epochs lie within $1.5\,\sigma$ from it. When removing the epochs $175$ and $229$, the RMS decreases to 88 seconds. 

Considering all the transit times, the linear fit has $\chi^{2}_{red}=1.4$, and the second degree polynomial has $\chi^{2}_{red}=1.39$ . However, we chose the linear fit as it suppose a simpler model.

\begin{table}
\caption{Transit mid-times for WASP-77Ab}
\label{times_wasp77}
\centering
\begin{tabular}{cccc}
\hline \hline
Epoch & Transit mid-time & TTV & References\\
      & (${\rm BJD_{TDB}}$) & (min) &  \\
\hline
-1140 & $2455870.45054$ & $-2.0\pm1.5$ & 1 \\
-845 & $2456271.65888$ & $-2.0\pm1.4$ & 2 \\
-659 & $2456524.62617$ & $0.8\pm1.8$ & This work\footnote{a} \\
-606 & $2456596.70591$ & $-1.7\pm1.5$  & This work\footnote{a}  \\
-92 & $2457295.7626$  & $1.2\pm1.3$& This work \\
-89 & $2457299.84119$ &  $-1.0\pm1.3$ & This work \\
175 & $2457658.88744$ & $-2.8\pm1.5$ & This work\\
177 & $2457661.60987$   & $0.6\pm1.7$ & This work\footnote{a} \\
183 & $2457669.77054$ &  $1.4\pm1.4$  & This work  \\
229 & $2457732.33382$ & $4.2\pm1.4$& This work\footnote{a} \\
447 & $2458028.8159$  & $-1.8\pm6.5$ & This work\\ 
\hline
\end{tabular}
%\tablebib{(1) \citet{Maxted2013}; (2) \cite{Turner2016}.}
%\tablefoot{
%\tablefoottext{a}{From the Exoplanet Transit Database (ETD)  \url{var2.astro.cz/ETD}}}
\end{table}
